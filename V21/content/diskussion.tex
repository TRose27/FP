\section{Diskussion}
\label{sec:Diskussion}
Die experimentell bestimmten Werte sind
\begin{align*}
  g_\text{F, 85} &= \num{0.563\pm0.027} \\
  g_\text{F, 87} &= \num{0.366\pm0.011} \\
  I_\text{85} &= \num{1.28\pm0.08} \\
  I_\text{87} &= \num{2.23\pm0.08} \\
  \Delta U_\text{85} &= \SI{1.26\pm0.07e-30}{\joule} \\
  \Delta U_\text{87} &= \SI{7.39\pm0.25e-31}{\joule} \\
  B_\text{Erde, hor.} &= \SI{12.7\pm2.6}{\micro\tesla} \\
  B_\text{Erde, vert.} &= \SI{34.6}{\micro\tesla} \\
  \left(\frac{\ce{^{87}Rb}}{\ce{^{85}Rb}}\right) &= \frac{1}{2}.
\end{align*}
Die bestimmten Landé Faktoren haben eine Abweichung von $\SI{12.6}{\percent}$ für Rubidium 85 und $\SI{9.8}{\percent}$ für Rubidium 87. Dies lässt auf eine gute Messgenauigkeit
schließen. Der quadratische Zeeman-Effekt konnte als Fehlerquelle ausgeschlossen werden, da er zu klein ist um die Messung signifikant zu beeinträchtigen.
Weitere mögliche Fehlerquellen sind eine unzureichende Kompensation des Erdmagnetfeldes und eventuelle Feldinhomogenitäten in den magnetischen Feldern der Helmholtzspulen,
trotzdem konnte das Verfahren des optischen Pumpens erfolgreich angewendet und die Theorie der Zeeman Aufspaltung mit hinreichender Genauigkeit verifiziert werden.\\
Die Komponenten des Erdmagnetfeldes sind im Vergleich zu typischen Werten von $B_\text{Erde, hor.} = \SI{20}{\micro\tesla}$ und $B_\text{Erde, vert.} = \SI{44}{\micro\tesla}$
\cite{erdfeld} circa $\SI{10}{\micro\tesla}$ zu niedrig, was vermutlich auf die Inhomogenität des Erdmagnetfeldes sowie Störungen durch äußere Einflüsse zurückzuführen ist. \\
Das Isotopengemisch weicht mit $\SI{28.2}{\percent}$ deutlich vom natürlichen Zustand ab, weshalb es äußerst unwahrscheinlich ist, dass natürliches Rubidium verwendet wurde, jedoch könnte
die Abweichung auch eine Folge der Messmethode ist, da das Amplitudenverhältnis aus einem Foto des Schirms des Oszilloskops bestimmt wurde, was eine geringe Genauigkeit bedingt.
Um qualifiziertere Aussagen über das Isotopengemisch treffen zu können müsste deshalb eine erneute genauere Messung durchgeführt werden.
