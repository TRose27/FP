\section{Durchführung}
\label{sec:Durchführung}
\begin{figure}[H]
    \centering
    \includegraphics[width=\textwidth]{images/aufbau.png}
    \caption{Schematischer Versuchsaufbau. \cite[16]{sample}}
    \label{fig:aufbau}
\end{figure}\FloatBarrier
Abbildung \ref{fig:aufbau} zeigt den schematischen Aufbau der Messaparatur. Die Dampfzelle ist mit dem zu untersuchenden Gas befüllt und wird durch das Licht der Rb-Dampflampe angeregt. 
Die rechtszirkulare Polarisation wird durch den Linearpolarisator und die $\lambda/4$-Platte erreicht. Der $D_1$-Filter lässt nur die $D_1$-Linie des Rb-Spektrums($\lambda=\SI{794,8}{\nano\meter}$)\cite[15]{sample}
durch und filtert alle anderen Wellenlängen heraus. Das Licht, welches durch die Dampfzelle hindurchtritt, wird von einem Si-Photoelement fokussiert und über einen Linearverstärker auf einem Oszilloskop 
dargestellt.

Zu Beginn wird der Strahlengang so justiert, dass er genau auf das Photoelement fällt und die Intensität maximal ist. Die ganze Apparatur ist so auszurichten, dass sie in Nord-Süd-Richtung 
steht, damit die horizontale Komponente des Erdmagnetfeldes, minimiert werden kann. Um die vertikale Komponente zu minimieren, wird das vertikale Magnetfeld so eingestellt, dass der auf dem 
Oszilloskop zu sehende Nullpeak, möglichst schmall wird.

Für die eigentliche Messung wird das Hochfrequenzfeld angelegt und von $\SI{100}{\kilo\hertz}$ bis $\SI{1}{\mega\hertz}$ in $\SI{100}{\kilo\hertz}$ Schritten variiert. Für die einzelnen 
eingestellten Frequenzen, wird mit dem angelegten Strom für die horizontale Sweep-Spule, die Resonanzstellen über das Oszilloskop ermittelt. Die eingestellten Stromstärken für die drei Magnetfelder werden notiert
und die nächste Frequenz untersucht. Ab einer Frequenz von ca. $\SI{200}{\kilo\hertz}$ ist es notwendig, ein zusätzliches horizontales Magnetfeld anzulegen, damit der Sweep-Feld-Bereich auf die 
Resonanzen verschoben werden kann.