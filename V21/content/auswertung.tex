\section{Auswertung}
\label{sec:Auswertung}
\FloatBarrier
\subsection{Bestimmung des Landé Faktors für Rubidium 85 und 87}
Die Messwerte für die Frequenz $\nu$, den Horizontalfeldspulenstrom $I_\text{H}$, den Sweepfeldspulenstrom $I_\text{S}$, sowie die mit der bekannten Formel für Helmholtzspulen
\begin{equation}
  B= \frac{\mu_0 \, 8 \, I \, N}{\sqrt{125} \, R}
  \label{eqn:hh}
\end{equation}
berechneten Magnetfeldstärken sind in Tabelle \ref{tab:1} aufgetragen. Die Werte für die Windungen und Radii der Spulen wurden aus \cite{sample} entnommen.
\begin{table}
  \centering
  \caption{Messwerte und Magnetfeldstärken.}
  \label{tab:1}
  \begin{tabular}[t]{c c c c c c c}
   \toprule
   & \multicolumn{3}{c}{Rubidium 85} & \multicolumn{3}{c}{Rubidium 87} \\
   \cmidrule(lr){2-4}\cmidrule(lr){5-7}
    {$\nu$ / $\si{\kilo\hertz}$} & {$I_\text{S}$ / $\si{\milli\ampere}$} & {$I_\text{H}$ / $\si{\milli\ampere}$} & {$B$ / $\si{\micro\tesla}$} &{$I_\text{S}$ / $\si{\milli\ampere}$} & {$I_\text{H}$ / $\si{\milli\ampere}$} &{$B$ / $\si{\micro\tesla}$} \\
     \midrule
     \csvreader[no head,
     late after line=\\,
     late after last line=\\\bottomrule]%
     {data/tab.csv}{}%
     {$\num{\csvcoli}$ & $\num{\csvcolii}$ & $\num{\csvcoliii}$ & $\num{\csvcoliv}$ & $\num{\csvcolv}$ & $\num{\csvcolvi}$ & $\num{\csvcolvii}$} %
   \end{tabular}
   \FloatBarrier
 \end{table}
Die Magnetfeldstärken werden nun gegen die Frequenzen aufgetragen und mittels SciPy 1.1.0 \cite{scipy} mit der Funktion $y=ax+b$ linear gefittet.
Dies ist in Abbildung \ref{fig:fit} dargestellt.
Die bestimmten Parameter sind:
\begin{align*}
  a_{85} &= \SI{0.127\pm0.006}{\micro\tesla\per\kilo\hertz} \\
  b_{85} &= \SI{12.264\pm3.752}{\micro\tesla} \\
  a_{87} &= \SI{0.195\pm0.006}{\micro\tesla\per\kilo\hertz} \\
  b_{87} &= \SI{13.188\pm3.482}{\micro\tesla}
\end{align*}
Durch Vergleich mit Gleichung \eqref{eqn:} lässt sich $g_\text{F}$ folgendermaßen aus dem Parameter $a$ bestimmen:
\begin{equation}
  g_\text{F} = \frac{4 \pi \, m_e}{e \, a}
\end{equation}
So wurden folgende Landé Faktoren bestimmt:
\begin{align*}
  g_\text{F, 85} &= \num{0.563\pm0.027} \\
  g_\text{F, 87} &= \num{0.366\pm0.011}.
\end{align*}
Die theoretischen Werte betragen $\sfrac{1}{2}$ für Rubidium 85 und $\sfrac{1}{3}$ für Rubidium 87 \cite{gF}, woraus sich die relativen Abweichungen zu $\SI{12.6}{\percent}$ und $\SI{9.8}{\percent}$ ergeben.
Aus den so ermittelten Landé Faktoren lässt sich auch der Kernspin der Rubidium Isotope mit Gleichung \eqref{eqn:} bestimmen.
So wurde für Rubidium 85 ein Kernspin von $\sfrac{5}{2}$ und für Rubidium 87 ein Kernspin von $\sfrac{3}{2}$ bestimmt.

\begin{figure}
  \centering
  \includegraphics{build/auswertung.pdf}
  \caption{Die grafische Darstellung der Messdaten, sowie die Fits für Rubidium 85 und 87.}
  \label{fig:fit}
\end{figure}
\FloatBarrier
\subsection{Bestimmung des Erdmagnetfeldes}
Da bei einer Frequenz von 0 auch der Spulenstrom 0 sein muss, lassen sich die Achsenabschnitte der Fits mit der Horizontalkomponente des Erdmagnetfeldes identifizieren.
Diese ist dann im Mittel
\begin{equation*}
B_\text{Erde, hor.} = \SI{12.7\pm2.6}{\micro\tesla}.
\end{equation*}
Die vertikale Komponente des Erdmagnetfeldes lässt sich aus dem gemessenem Vertikalspulenstrom von $\SI{226}{\milli\ampere}$ und Gleichung \eqref{eqn:hh} zu
\begin{equation*}
  B_\text{Erde, vert.} = \SI{34.6}{\micro\tesla}
\end{equation*}
bestimmen.

\subsection{Bestimmung des Isotopengemisches}
\FloatBarrier
Aus der in Abbildung \ref{fig:resonanz} zu sehenden Resonanzkurve wurde das Amplitudenverhältnis zu $\sfrac{1}{2}$ bestimmt. Da dieses gleich dem Isotopenverhältnis ist,
ist dies auch das Isotopenverhältnis zwischen Rubidium 87 und 85. Der natürlich vorkommende Wert \cite{Rb} beträgt
\begin{equation*}
  \left(\frac{\ce{^{87}Rb}}{\ce{^{85}Rb}}\right)_\text{natürlich} = 0,39 .
\end{equation*}
Somit beträgt die relative Abweichung von der natürlichen Isotopenverteilung $\SI{28.2}{\percent}$.
\begin{figure}
  \centering
  \includegraphics[scale=0.25]{images/kurve.jpeg}
  \caption{Gemessene Resonanzkurve bei $\SI{100}{\kilo\hertz}$.}
  \label{fig:resonanz}
\end{figure}
\FloatBarrier
