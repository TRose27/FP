\section{Theory}
\label{sec:Theorie}
Quantum particles like electrons possess a measureable angular momentum even if the ground angular momentum of the atomic shell they are in is zero or if they are free.
This is called the electron spin. The spin obeys the laws of quantum mechanics wich state that angular momentums are not continuous, but discrete values.
Since electrons are charged, their angular momentums induce magnetic moments.
The quantization of those magnetic moments can be derived from the wave function of an one-electron atom
  \begin{equation}
    \psi_{n,l,m}(r,\vartheta,\varphi)=R_{n,l}(r)\uptheta_{l,m}(\vartheta)\upphi(\varphi)=\frac{R_{n,l}(r)\uptheta_{l,m}\symup{e}^{im\varphi}}{\sqrt{2\pi}},
  \end{equation}
which leads to the following equation for the z-component of the magnetic moment
\begin{equation}
 \mu_z = - \frac{1}{2} \frac{e_0}{m_0} m \hbar = \mu_{\mathrm{B}} m
 \label{eq:muz}
\end{equation}
with the elementary charge $e_0$, the electron mass $m_0$, the reduced planck constant $\hbar$, the Bohr magneton $\mu_B$ and the magnetic quantum number $m$.
When put into a homogenous magnetic field $\vec{B}$ along the z-axis the angular momentums split into $2l+1$ discrete directions which are described $m$, showing its relation with
the azimuthal quantum number $l$.
Those directions also correspond to energy levels of equal distance. This is called the Zeeman effect.

As it is observed in the Stern-Gerlach experiment the electron spin splits into two energy levels which can be described by the spin quantum number $s=\frac{1}{2}$ and its
direction quantum number $m_S=\pm\frac{1}{2}$. But when plugged into \eqref{eq:muz} a factor $g$ needs to be added to match the measured results.

This g-factor can be measured by putting free electrons into a homogenous magnetic field, casuing them to split into two energy levels which differ by
\begin{equation}
  \Delta E = g \mu_{\mathrm{B}} B.
\end{equation}
This difference in energy is now given to the system in form of electromagnetic waves leading to electrons with lower energy changing their direction.
Since this is not a stable state the electrons dispense the energy in complex ways to get back to the lower energy level.
This whole process is called electron spin resonance and with the energy of the used radiation, the following connection can be found:
\begin{equation}
  h \nu = g \mu_{\mathrm{B}} B.
  \label{eq:hnu}
\end{equation}
This is also depicted in figure \ref{fig:esr}.
The homogenous magnetic field is created by a Helmholtz coil, for which the magnetic field correlates with the used current:
\begin{equation}
  B(I)=\frac{8\mu_0 n I}{\sqrt{125}\cdot r}.
  \label{eq:hh}
\end{equation}

\begin{figure}
  \centering
  \includegraphics{images/esr.png}
  \caption{Graphic representation of electron spin resonance.\cite{anleitung}}
  \label{fig:esr}
\end{figure}
