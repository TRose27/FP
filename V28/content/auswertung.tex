\section{Evaluation}
\label{sec:Evaluation}
To determine the g-factor, formula \eqref{eq:hnu} can be converted to
\begin{equation}
  g = \frac{h \nu}{\mu_{\mathrm{B}} B},
  \label{eqn:g}
\end{equation}
with the Planck constant $h = \SI{6.626e-34}{\joule\second}$ \cite{h} and the Bohr magneton $\mu_{\mathrm{B}} = \SI{927.4e-26}{\joule\per\tesla}$ \cite{mub}.
The used frequencies and currents as well as the strength of the homogenous magnetic field, which was calculated using \eqref{eq:hh}, with $n=156$ and $r=\SI{0.1}{\metre}$, are shown in
table \ref{tab:g-factor}.
\begin{table}
  \centering
  \caption{The determined g-factor for different frequencies $\nu_{\mathrm{e}}$.}
  \label{tab:g-factor}
  \begin{tabular}{c c c c}
    \toprule
    $\nu_{\mathrm{e}} \,/\, \si{\mega\hertz}$ & $I \,/\, \si{\milli\ampere}$ & $B \,/\, \si{\milli\tesla}$ & $g$\\
    \midrule
    \csvreader[respect sharp=true,
    late after line=\\,
    late after last line=\\\bottomrule]{data/datatab.csv}{}
    {\num{\csvcoli} & \num{\csvcolii} & \num{\csvcoliii} & \num{\csvcoliv}}
  \end{tabular}
\end{table}\FloatBarrier
The average value for the g-factor is $g = \num{2.22(15)}$.
