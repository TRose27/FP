\section{Durchführung}
\label{sec:Durchführung}
Zunächst wird eine Messapparatur wie in Abbildung \ref{fig:aufbau} zu sehen aufgebaut.
\begin{figure}
    \centering
    \includegraphics{images/aufbau.png}
    \caption{Schematische Darstellung des Versuchaufbaus.\cite{sample}}
    \label{fig:aufbau}
\end{figure}
Diese besteht zunächst einmal aus einer Halogen-Lampe als Lichtquelle und einer Linse zur Fokussierung.
Dann durchläuft der Lichtstrahl einen auf eine feste Frequenz eingestellten Lichtzerhacker, an welchen sich
der drehbare Polarisator, in diesem Fall ein Glan-Thompson-Prisma, mitsamt Goniometer anschließt. Dann durchläuft der Lichstrahl die Probe welche sich in
einem, durch zwei Elektromagneten erzeugtem, Magnetfeld befindet. Anschließend wird die zu untersuchende Wellenlänge mit einem Interferenzfilter
herausgefiltert und der Strahl durch ein weiteres Glan-Thompson-Prisma in zwei Strahlanteile aufgespalten. Die Intensität dieser Strahlen wird durch
eine Sammellinse mit anschließendem PbS Photowiderstand gemessen und einmal über ein regelbares RC-Glied und einmal nur über einen Widerstand abgegriffen. Diese
beiden Spannungen werden nun auf die Eingänge eines Differenzverstärkers gegeben, wodurch nur ihre Differenz übrig bleibt. Abschließend wird noch ein
Selektivverstärker mit der selben Filterfrequenz wie die des Lichtzerhackers angeschlossen, um Störspannungen rauszufiltern. Das so entstandene monofrequente
Signal wird dann auf einem Oszilloskop ausgewertet.


Zur Kalibrierung des Gerätes wird nun der Interferenzfilter entfernt und geprüft ob der Lichtstrahl auf beide Photowiderstände trifft, und ob bei geeigneter
Stellung des ersten Polarisators ein Lichtsrahl nicht mehr sichtbar und der andere maximal hell ist. Anschließend wird die Güte des Selektivverstärkers maximal
eingestellt und die Verstärkung sowie die Zeitkonstante des RC-Glieds so eingeregelt, dass sich bei geeigneter Polarisatorstellung eine minimale Signalamplitude
ergibt.

Zur Messung wird der Elektromagnet mit einem möglichst hohen Gleichstrom versorgt und die Probe in den Luftspalt eingebracht. Nun wird der Polarisator so eingestellt,
dass sich eine minimale Signalamplitude ergibt, und der Winkel vom Goniometer notiert. Anschließend wird der Gleichstrom am Magneten, und damit das Magnetfeld,
umgepolt und die Messung wiederholt. Dies wird einmal für reines Galliumarsenid und einmal für n-dotiertes Galliumarsenid mit Wellenlängen zwischen 1 und 3 $\si{\micro\metre}$
durchgeführt.

Da durch die Umpolung des Feldes der Rotationswinkel doppelt gemessen wird, berechnet sich der eigentliche Rotationswinkel folgendermaßen
\begin{equation}
  \theta = \frac{1}{2}(\theta_1 - \theta_2)
  \label{eqn:theta}
\end{equation}

Abschließend wird noch mit einer Hallsonde das Magnetfeld innerhalb und zwischen den Elektromagneten vermessen.
