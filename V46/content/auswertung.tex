\section{Auswertung}
\label{sec:Auswertung}

\subsection{Ermittlung der maximalen Kraftflussdichte des Magnetfeldes}
Die mit der Hallsonde bestimmten Werte für die magnetische Flussdichte, in Abhängigkeit von der Position, sind in Tabelle \ref{tab:Bz} zu finden.
\begin{table}[H]
    \centering
    \caption{Die Messwerte für die magnetische Flussdichte in der Umgebung der Probe.}
    \label{tab:Bz}
    \begin{tabular}{c c}
        \toprule
        $z\,/\,\mathrm{mm}$ & $B\,/\,\mathrm{mT}$\\
        \midrule
        \csvreader[respect sharp, late after line=\\, late after last line=\\\bottomrule]{data/Bz.csv}{}{\num{\csvcoli} & \num{\csvcolii}}
    \end{tabular}
\end{table}\FloatBarrier
Der dazugehörige Graph ist in Abbildung \ref{fig:Bz} zu sehen. Es wird angenommen, dass der maximale Wert des Graphen, der magnetischen Flussdichte am Ort der Probe 
entspricht.
\begin{figure}[H]
    \centering
    \includegraphics{build/Bz.pdf}
    \caption{Magnetische Flussdichte in Abhängigkeit der Postion zur Bestimmung der maximalen Flussdichte.}
    \label{fig:Bz}
\end{figure}\FloatBarrier
Das abgelesene Maximum und die damit folgene maximale Flussdichte am Ort der Probe lautet
\begin{equation*}
    B_0 = \SI{414}{\milli\tesla}.
\end{equation*}




\subsection{Bestimmung der effektiven Masse}
Die verwendeten Proben mit ihrer Dicke und Dotierung sind in Tabelle \ref{tab:proben} zu sehen.
\begin{table}[H]
    \centering
    \caption{Die Daten zu den verwendeten Proben.}
    \label{tab:proben}
    \begin{tabular}{c c c}
        \toprule
        Bezeichnung & Dotierung $N\,/\,\frac{1}{\mathrm{cm^3}}$ & Dicke$\,/\,\mathrm{mm}$\\
        \midrule
        $\mathrm{GaAs}_{\mathrm{rein}}$    &        -        & 5,11\\
        $\mathrm{GaAs}_{\mathrm{dotiert}}$ & $\num{1,2e18}$  & 1,36\\
        \bottomrule
    \end{tabular}
\end{table}\FloatBarrier
Die gemessenen Winkel für die beiden Proben sind in der Tabelle \ref{tab:Messergebnisse} zu finden. Desweiteren sind in der Tabelle die Drehwinkel für die 
beiden Proben nach Gleichung \eqref{eqn:} berechnet und durch die jeweilige Dicke der Probe geteilt, damit die Werte normiert sind. In der letzten Spalte sind die 
Differenzen der normierten Winekl zu sehen, um den Faraday-Effekt der Leitungselektronen zu erhalten.
\begin{table}[H]
 \centering
 \caption{Auflistung der gemessenen Winkel zu den jeweiligen Wellenlägen und die sich daraus ergebenen Polarisationswinkel.}
 \label{tab:Messergebnisse}
 \begin{tabular}{c c c c c c c c}
  \toprule
  $\lambda\,/\,\mathrm{\upmu m}$ & $\phi_{1,\mathrm{rein}}\,/\,\mathrm{rad}$ & $\phi_{2,\mathrm{rein}}\,/\,\mathrm{rad}$ & 
  $\theta_{\mathrm{rein}}\,/\,\frac{\mathrm{rad}}{\mathrm{mm}}$ & $\phi_{1,\mathrm{dot}}\,/\,\mathrm{rad}$ & $\phi_{2,\mathrm{dot}}\,/\,\mathrm{rad}$ 
  & $\theta_{\mathrm{dot}}\,/\,\frac{\mathrm{rad}}{\mathrm{mm}}$ & $\theta_{\mathrm{norm}}\,/\,\frac{\mathrm{rad}}{\mathrm{mm}}$\\
  \midrule
  \csvreader[respect sharp, late after line = \\, late after last line=\\\bottomrule]{data/daten2.csv}{}
  {\num{\csvcoli} & \num{\csvcolii} & \num{\csvcoliii} & \num{\csvcoliv} & \num{\csvcolv} & \num{\csvcolvi} & \num{\csvcolvii} & \num{\csvcolviii}}
 \end{tabular}
\end{table}\FloatBarrier
Die Gleichung \eqref{eqn:} lässt sich nach dem normierten Drehwinkel umstellen:
\begin{equation}
    \frac{\theta}{L} = \theta_{\mathrm{norm}} = \frac{e_0^3}{8\pi^2\epsilon_0c^3}\cdot\frac{1}{{m^*}^2}\cdot\frac{NB}{n}\lambda^2\;.
    \label{eqn:gerade}
\end{equation}
Da auf der rechten Seite der Gleichung nur die Wellenlänge $\lambda$ nicht konstant ist, gibt es einen linearen Zusammenhang zwischen dem Quadrat der Wellenlänge und dem 
normierten Drehwinkel $\theta_{\mathrm{norm}}$. Es wird daher eine lineare Regression mit der Form 
\begin{equation*}
    \theta_{\mathrm{norm}}(\lambda^2) = a\cdot \lambda^2 + b 
\end{equation*}
verwendet. Die Messwerte und die lineare Regression sind in Abbildung \ref{fig:Darstellung} zu sehen.
\begin{figure}[H]
    \centering
    \includegraphics{build/Messergebnisse.pdf}
    \caption{Die Differenz der normierten Drehwinkel gegen die Wellenlänge zum quadrat.}
    \label{fig:Darstellung}
\end{figure}\FloatBarrier
Die Parameter für die Ausgleichsgerade sind 
\begin{align*}
    a &= (2,08\pm0,26)\cdot 10^{14}\,\frac{\mathrm{rad}}{\mathrm{m^3}}\;,\\
    b &= (726\pm117)\,\frac{\mathrm{rad}}{\mathrm{m}}\;.
\end{align*}
Aus Gleichung \eqref{eqn:gerade} folgt nun, dass die Steigung $a$ folgende Gleichung erfüllt:
\begin{equation*}
    a = \frac{e_0^3}{8\pi^2\epsilon_0c^3}\cdot\frac{1}{{m^*}^2}\cdot\frac{NB}{n}\;.
\end{equation*}
Nach der effektiven Masse $m^*$ umgestellt, ergibt sich ein Wert von 
\begin{equation}
    m^* = \sqrt{\frac{e_0^3}{8\pi^2\epsilon_0c^3}\cdot\frac{1}{a}\cdot\frac{NB}{n}} = \SI{1,26\pm0,08e-32}{\kilo\gram}\;.
\end{equation}
Für den Brechungsindex wird dabei der Wert $n=3,3$\cite{n} verwendet.
Der Fehler berechnet sich nach der Gauß'schen Fehlerfortpflanzung
\begin{equation*}
    \increment f=\sqrt{\sum_{i=1}^N \left(\frac{\partial f}{\partial x_i}\right)^2 \cdot (\increment x_i)^2}.
\end{equation*}
Mit der fehlerbehafteten Größe $a$ ergibt sich für die effektive Masse $m^*$ mit 
\begin{equation*}
    k=\frac{e_0^3}{8\pi^2\epsilon_0c^3}\cdot\frac{NB}{n}
\end{equation*}
der Fehler nach
\begin{align*}
    \increment m^* &= \sqrt{\left(\frac{\mathrm{d}m^*}{\mathrm{d}a}\right)^2 \cdot \increment a^2}\;,\\
                   &= \sqrt{\left(-\frac{1}{2}\left(\frac{k}{a}\right)^{-\frac{1}{2}}(ka^{-2})\right)^2 \cdot \increment a^2}\;.
\end{align*}