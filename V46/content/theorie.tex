\section{Theorie}
\label{sec:Theorie}
\subsection{Die effektive Masse}
Zur Beschreibung von Halbleitern wird ihre komplexe Bandstruktur oftmals durch Näherungen beschrieben.
Dies geschieht, wie in Abbildung \ref{fig:meff} zu sehen, durch anlegen einer Parabel an das Leitfähigkeitsband.
\begin{figure}
    \centering
    \includegraphics{images/meff.png}
    \caption{Parabolische Näherung am Leitungsband eines Halbleiters.\cite{sample}}
    \label{fig:meff}
\end{figure}
Wird nun die Dispersionsrelation $E(\vec{k})$ der Elektronen in diesem Band als Taylorreihe in $k=0$ entwickelt und mit der genäherten parabolischen Dispersionsrelation
verglichen lässt sich folgende Größe mit der Dimension einer Masse definieren:
\begin{equation}
  m_i^* = \frac{\hbar^2}{\left\{\frac{\partial^2 E}{\partial k_i^2}\right\}_{k=0}} .
\end{equation}
Diese wird als effektive Masse des Leitungselektrons bezeichnet. Sie kann in die verschiedenen Raumrichtungen variieren, oder aber, im Falle eines
hoch symmetrischen Halbleiters, gleich sein. In diesem Fall lassen sich die Elektronen im Leitungsband durch Nutzung der effektiven Masse als freie Teilchen beschreiben.
Daraus ergibt sich weiterhin, dass beim Anlegen eines äußeren elektrischen oder magnetischen Feldes newtonsche Mechanik angewendet werden kann:
\begin{equation}
  m^* \cdot \vec{b} = \vec{F} .
\end{equation}

\subsection{Zirkuläre Doppelbrechung}
\label{sec:brechung}
Sogenannte optisch aktive Kristalle verfügen über die Fahigkeit die Polarisationsebene von einfallendem Licht zu drehen.
Dies ist in Abbildung \ref{fig:brechung} dargestellt.
\begin{figure}
    \centering
    \includegraphics{images/brechung.png}
    \caption{Darstellung der zrikulären Doppelbrechung optisch aktiver Kristalle.\cite{sample}}
    \label{fig:brechung}
\end{figure}
Zur quantitativen Beschreibung dieser zirkulären Doppelbrechung wird die Wellenfunktion des einfallenden Lichtes in einen rechts- und einen linkszirkulären Anteil aufgeteilt
und somit eine Gleichung für den Drehwinkel $\theta$ aufgestellt. Nun werden noch die durch Lichteinfall induzierten Dipole im Festkörper und die dadruch entstehende
Polarisation des Kristall berücksichtigt woraus letzendlich folgende Gleichung
\begin{equation}
  \theta = \frac{L \omega}{2 c n} \chi_\text{xy},
\end{equation}
mit der Kirstalllänge $L$, der Frequenz des Lichtes $\omega$, der Lichtgeschwindigkeit $c$, dem Brechungsindex $n$ und der $xy$ Kompenente des dielektrischen
Suszeptibilitätstensors des Kristalles, folgt.

\subsection{Der Faraday Effekt}
Beim Faraday Effekt handelt es sich um die Fähigkeit optisch inaktiver Festkörper die Polarisationsebene von einfallenden Licht zu drehen, sofern ein Magnetfeld
parallel zur Strahlrichtung angelegt wird. Dies geht auf die Wirkung des Magnetfeldes auf die Elektronen im Festkörper zurück.
Zur Beschriebung dieses Phänomens wird analog zu \ref{sec:brechung} vorgegangen, jedoch muss das äußere Magnetfeld berücksichtigt werden.
So lassen sich zwei Gleichungen für den Drehwinkel herleiten. Für den Fall gebundener Elektronen gilt
\begin{equation}
  \theta_\text{gebunden} =\frac{2 \pi^2 e_0^3 c}{\epsilon_0}\frac{1}{m^2}\frac{1}{\lambda^2 \omega_0^4}\frac{NBL}{n}
  \label{eqn:gebunden}
\end{equation}
mit der Elementarladung $e_0$, der Lichtgeschwindigkeit $c$, der elektrischen Feldkonstante $\epsilon_0$, der Elektronenmasse $m$, der Wellenlänge des einfallenden
Lichtes $\lambda$, der Resonanzfrequenz des Festkörpers $\omega_0$, der Elektronendichte $N$, der magnetischen Flussdichte $B$, der Länge des Festkörpers $L$ und dem
Brechungsindex $n$.
Für den Fall quasifreier Elektronen gilt
\begin{equation}
  \theta_\text{frei} = \frac{e_0^3}{8 \pi^2 \epsilon_0 c^3}\frac{\lambda^2}{m^2}\frac{NBL}{n}.
  \label{eqn:frei}
\end{equation}
Unter den in diesem Versuch geschaffenen Bedingungen behält Gleichung \eqref{eqn:frei}, auch wenn $m$ durch die effektive Masse $m^*$ ersetzt wird, ihre Gültigkeit.
