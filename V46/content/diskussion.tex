\section{Diskussion}
\label{sec:Diskussion}
Der Literaturwert für die effektive Masse von GaAs lautet $m^*_{\mathrm{lit}} = 0,067m_0$\cite{m*}, mit der Ruhemasse eines freien Elektrons 
$m_0=\SI{9,11e-31}{\kilo\gram}$. Der experimentelle bestimmte Wert von $m^*_{\mathrm{exp}} = \SI{9,06e-33}{\kilo\gram}=0,0099m_0$ entspricht daher einer Abweichung 
von $\SI{85,22}{\percent}$ vom Literaturwert.

Die hohe Abweichung lässt sich durch die ungenaue Messung erklären. 
Es war nicht möglich das Messsignal vernünftig zu nullen. Daraus folgte, dass es bei jeder Messung zwei bis drei einstellbare Winkel gab, bei denen das 
Oszilloskop ein Minimum angezeigt hat. Welches von diesen Minima das tatsächlich kleinste war, ließ sich nicht sagen. Die Folge daraus sind die negativen 
normierten Winkel, welche bei der linearen Regression nicht berücksichtigt wurden. Deshalb wurden nur 4 Messwerte für die Ausgleichsgerade verwendet, wodurch 
die bestimmten Parameter der Gerade ungenau sind.\\
Die Ungenauigkeit der Messung zeigt sich auch bei der linearen Regression in Abbildung \ref{fig:Darstellung}. Nach der Theorie sollte der y-Achsenabschnitt der 
Ausgleichsgerade im Ursprung liegen. Da dies aber nicht der Fall ist, lässt sich anhand der Höhe des y-Achsenabschnittes sagen, dass die Messwerte im Allgemeinen 
zu groß sind.\\
Ein weiterer Fehler hängt mit dem Magnetfeld zusammen. Da die Messung der Magnetfeldstärke erst am Ende des Versuches stattfand, ist es Wahrscheinlich, dass 
die Magnetfeldstärke am Anfang des Versuches noch stärker war, da sich im Laufe der Zeit die Apparatur erhitzt und die Magnetfeldstärke damit kleiner wird.




