\section{Diskussion}
\label{sec:Diskussion}
Für die Drehschieberpumpe wurden die Saugvermögen
\begin{align*}
  S_1 &= \SI{1.04\pm0.08}{\litre\per\second} \\
  S_2 &= \SI{0.57\pm0.06}{\litre\per\second}
\end{align*}
durch Evakuierungsmessung, und die Saugvermögen
\begin{align*}
  S_3 &= \SI{0.53\pm0.12}{\litre\per\second} \\
  S_4 &= \SI{0.92\pm0.20}{\litre\per\second} \\
  S_5 &= \SI{1.32\pm0.28}{\litre\per\second} \\
  S_6 &= \SI{1.37\pm0.29}{\litre\per\second}
\end{align*}
durch Leckratenmessung bestimmt.
Für die Turbopumpe wurden die Saugvermögen
\begin{align*}
  S_1 &= \SI{6.4\pm0.8}{\litre\per\second}   \\
  S_2 &= \SI{0.245\pm0.04}{\litre\per\second}
\end{align*}
durch Evakuierungsmessung, und die Saugvermögen
\begin{align*}
  S_3 &= \SI{9.1\pm1.2}{\litre\per\second}  \\
  S_4 &= \SI{19.2\pm2.5}{\litre\per\second}  \\
  S_5 &= \SI{20.5\pm2.6}{\litre\per\second}  \\
  S_6 &= \SI{16.4\pm2.1}{\litre\per\second}
\end{align*}
durch Leckratenmessung bestimmt.
Die Herstellerangaben für die Saugvermögen betragen $\SI{1.1}{\litre\per\second}$ für die Drehschieberpumpe und $\SI{77}{\litre\per\second}$ für die
Turbopumpe \cite{leitfaden}. Bei der Drehschieberpumpe sind wie in Abbildung \ref{fig:drehvgl} zu sehen die Saugvermögen bei niedrigen Drücken deutlich geringer als bei höheren. Dies geht darauf
zurück, dass das Saugvermögen der Pumpen auch vom Druck im Rezipienten abhängig ist. Bei den Saugvermögen $S_1$, $S_5$ und $S_6$ liegt jedoch die Herstellerangabe innerhalb der
Messgenauigkeit und kann somit für ausreichend hohe Drücke bestätigt werden. \\\\
Bei der Turbopumpe sind ebenfalls geringere Saugvermögen bei niedrigen Drücken gemessen worden, da auch hier eine Druckabhängigkeit besteht. Jedoch fällt der Unterschied deutlich
größer aus als bei der Drehschieberpumpe. Es ist auch zu beobachten, dass die durch Leckratenmessungen bestimmten Werte deutlich größer sind als jene, welche
durch die Evakuierungsmessung bestimmt wurden, und dass alle Werte deutlich unter der Herstellerangabe liegen. Die geht vermutlich auf den in Gleichung \eqref{eqn:seff} dargestellten Zusammenhang
zwischen dem effektiven Saugvermögen und dem Leitwert zurück. Das gemessene Saugvermögen ist das effektive Saugvermögen und folglich niedriger als die Herstellerangabe.
Die größe des Unterschiedes ist dabei abhängig vom Leitwert, welcher unter anderem wie in Gleichung \eqref{eqn:leitwertmolekular} zu sehen vom Querschnitt des Rohres abhängt, welcher wie
in Kapitel \ref{sec:Durchführung} zu sehen direkt vor dem Ansaugstutzen der Turbopumpe durch eine Querschnittsverengung halbiert wird.
Eine weitere Fehlerquelle sind virtuelle Lecks, welche vom verwendeten Pumpstand abhängig sind und bei sinkenden Drücken an Relevanz dazu gewinnen.
Außerdem wird das vom Hersteller angegebene Saugvermögen
nur unter optimalen Bedingungen, welche zum Beispiel die Nutzung von purem Stickstoff und die Minimierung von realen und virtuellen Lecks darstellen, erreicht.\\\\
Abschließend lässt sich die Herstellerangabe bei der Drehschieberpumpe verifizieren, bei der Turbopumpe jedoch aufgrund von unbekannten Leitwerten und Bedingungen nicht.
