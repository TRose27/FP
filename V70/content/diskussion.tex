\section{Diskussion}
\label{sec:Diskussion}
Die gemessenen Saugvermögen betragen
\begin{align*}
  S_1 &= \SI{1.04\pm0.08}{\litre\per\second} \\
  S_2 &= \SI{0.57\pm0.06}{\litre\per\second} \\
  S_3 &= \SI{0.53\pm0.12}{\litre\per\second} \\
  S_4 &= \SI{0.92\pm0.20}{\litre\per\second} \\
  S_5 &= \SI{1.32\pm0.28}{\litre\per\second} \\
  S_6 &= \SI{1.37\pm0.29}{\litre\per\second}
\end{align*}
für die Drehschieberpumpe und
\begin{align*}
  S_1 &= \SI{7.0\pm0.8}{\litre\per\second}   \\
  S_2 &= \SI{0.27\pm0.04}{\litre\per\second} \\
  S_3 &= \SI{10.0\pm1.2}{\litre\per\second}  \\
  S_4 &= \SI{21.1\pm2.6}{\litre\per\second}  \\
  S_5 &= \SI{22.5\pm2.8}{\litre\per\second}  \\
  S_6 &= \SI{18.0\pm2.2}{\litre\per\second}
\end{align*}
für die Turbopumpe.
Die Herstellerangaben für die Saugvermögen betragen $\SI{1.1}{\litre\per\second}$ für die Drehschieberpumpe und $\SI{77}{\litre\per\second}$ für die
Turbopumpe \cite{leitfaden}. Bei der Drehschieberpumpe sind wie in \ref{sec:vgl} zu sehen die Saugvermögen bei niedrigen Drücken deutlich geringer als bei höheren. Dies geht darauf
zurück, dass das Saugvermögen der Pumpen auch vom Druck im Rezipienten abhängig ist. Bei den Saugvermögen $S_1$, $S_5$ und $S_6$ liegt jedoch die Herstellerangabe innerhalb der
Messgenauigkeit und kann somit für ausreichend hohe Drücke bestätigt werden.
Bei der Turbopumpe sind ebenfalls geringere Saugvermögen bei niedrigen Drücken gemessen worden, da auch hier eine Druckabhängigkeit besteht. Jedoch fällt der Unterschied deutlich
größer aus als bei der Drehschieberpumpe. Dies ist wahrscheinlich darauf zurück zu führen, dass die Turbopumpe für bestimmte mittlere freie Weglängen der Teilchen optimiert ist und
somit nur in diesen Bereichen eine großes Saugvermögen auftritt. Es ist auch zu beobachten, dass die durch Leckratenmessung bestimmten Werte deutlich größer sind als jene, welche
duch die Evakuierungsmessung bestimmt wurden, und dass alle Werte deutlich unter der Herstellerangabe liegen. Die geht vermutlich auf den in \eqref{eqn:seff} dargestellten Zusammenhang
zwischen dem effektiven Saugvermögen und dem Leitwert zurück. Da das gemessene Saugvermögen das effektive Saugvermögen darstellt, ist es logisch ,dass die Herstellerangabe größer als die
gemessen Werte ist. Die größe des Unterschiedes ist dabei abhängig vom Leitwert, welcher unter anderem wie in \eqref{eqn:leitwertmolekular} vom Querschnitt des Rohres abhängt, welcher wie
in \ref{sec:Durchführung} zu sehen direkt vor dem Ansaugstutzen der Turbopumpe durch eine Querschnittsverengung massiv verkleinert wird. Außerdem wird das vom Hersteller angegebene Saugvermögen
nur unter optimalen Bedingungen erreicht.
Abschließend lässt sich die Herstellerangabe bei der Drehschieberpumpe verifizieren, bei der Turbopumpe jeeoch aufgrund von unbekannten Leitwerten und Bedingungen nicht.
Desweiteren ist es ersichtlich, dass eben diese Bedingungen bei molekularen Strömungen und auch bei Vakua höherer Qualitäten einen größeren Einfluss auch das Saugvermögen haben.
