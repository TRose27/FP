\section{Auswertung}
\label{sec:Auswertung}
\subsection{Volumenbestimmung}
Die im folgenden verwendeten Zahlen für die Volumina sind in den Abbildungen \ref{fig:ps1} bis \ref{fig:ps3} zu sehen. Die Volumina der benutzten Bauteile sind in Tabelle \ref{tab:volumina}
aufgelistet. Zu vernachlässigende oder nicht vorhandene Volumina sind dabei als $0$ angegeben. Die Volumina 2 und 8 wurden dabei selber gemessen, die restlichen Werte wurden aus \cite{leitfaden} entnommen. Zur Bestimmung des Volumens von Bauteil 2 wurden folgende Werte gemessen:
\begin{align*}
  d_\text{innen}&= \SI{40\pm0.5}{\milli\metre} \\
  l_1 &= \SI{128.2\pm0.5}{\milli\metre}\\
  l_2 &= \SI{45\pm2}{\milli\metre}
\end{align*}
Das Volumen berechnet sich dann mit der Formel
\begin{equation*}
  V_2 = \pi \bigl(\frac{d_\text{innen}}{2}\bigr)^2 \cdot (l_1 + l_2).
\end{equation*}
und der Fehler nach \eqref{eqn:gaußerr} mit
\begin{equation*}
\Delta V_2 =  \sqrt{\frac{\pi^{2} d_{innen}^{4}}{16} \Delta_{l_{1}}^{2} + \frac{\pi^{2} d_{innen}^{4}}{16} \Delta_{l_{2}}^{2} + \frac{\pi^{2} d_{innen}^{2}}{4} \Delta_{d_{innen}}^{2} \left(l_{1} + l_{2}\right)^{2}}
\end{equation*}
zu dem in \ref{tab:volumina} angegebenen Wert. Bei Bauteil 8 wurden folgende Werte und, da es sich um ein Kreuzstück handelt, Formeln genutzt:
\begin{align*}
  d_\text{innen}&= \SI{10\pm0.5}{\milli\metre} \\
  l_1 &= \SI{80\pm0.5}{\milli\metre}\\
  l_2 &= \SI{30\pm1}{\milli\metre} \\
  V_8 &= \pi \bigl(\frac{d_\text{innen}}{2}\bigr)^2 \cdot (l_1 + 2 l_2) \\
  \Delta V_8 &= \sqrt{\frac{\pi^{2} d_{innen}^{4}}{16} \Delta_{l_{1}}^{2} + \frac{\pi^{2} d_{innen}^{4}}{4} \Delta_{l_{2}}^{2} + \frac{\pi^{2} d_{innen}^{2}}{4} \Delta_{d_{innen}}^{2} \left(l_{1} + 2 l_{2}\right)^{2}}
\end{align*}
Bei den Messungen mit der Drehschieberpumpe ist das Ventil 6, sowie das in \ref{fig:ps2} zu sehende Ventil 4 verschlossen.
Die Addition der verbleibenden Volumina ergibt ein Gesamtvolumen von $\SI{11.1\pm0.8}{\litre}$.
Bei den Messungen mit der Trubopumpe sind diese Ventile offen, wodurch sich ein Gesamtvolumen von $\SI{11.2\pm0.8}{\litre}$ ergibt.
Der Fehler des Gesamtvolumens berechnet sich dabei wie folgt:
\begin{equation*}
    \Delta V_\text{ges}=\sqrt{\sum_{i=1}  (\increment V_i)^2}.
\end{equation*}
Dabei bezeichnet der Index $i$ die verwendeten Teile.
\begin{table}
  \centering
  \caption{Die Volumina der verwendeten Bauteile.}
  \label{tab:volumina}
  \begin{tabular}[t]{c c c}
   \toprule
    {Bauteil} & {Volumen (offen) / $\si{\litre}$} & {Volumen (geschlossen) / $\si{\litre}$} \\
     \midrule
     \csvreader[no head,
     late after line=\\,
     late after last line=\\\bottomrule]%
     {data/voltab.csv}{}%
     {$\num{\csvcoli}$ & $\num{\csvcolii} \pm \num{\csvcoliii}$ & $\num{\csvcoliv} \pm \num{\csvcolv}$ }%
   \end{tabular}
 \end{table}
 \FloatBarrier
\subsection{Drehschieberpumpe}
\subsubsection{Evakuierungsmessung}
\FloatBarrier
\label{sec:drehevak}
Zur Bestimmung des Saugvermögens aus der Evakuierungsmessung wird Gleichung \eqref{eqn:pkurve} umgestellt und logarhithmiert, so dass sich folgende Gleichung ergibt:
\begin{equation}
  \ln \Bigl(\frac{p(t)-p_E}{p_0-p_E}\Bigr) = - t \cdot \frac{S}{V}.
  \label{eqn:lnpkurve}
\end{equation}
Die Messwerte, sowie die mit \eqref{eqn:mittelwert} und \eqref{eqn:mittelwerterr} bestimmten Zeiten befinden sich in Tabelle \ref{tab:drehevak}.
Zur Berechnung von $  \ln \Bigl(\frac{p(t)-p_E}{p_0-p_E}\Bigr)$ wurden der Atmosphärendruck $p_0=\SI{1013}{\milli\bar}$ und der gemessene Enddruck $p_E=\SI{4e-2}{\milli\bar}$ verwendet.
Die Fehler aller gemessen Drücke betragen, gemäß der aus \cite{leitfaden} entnommenen Messungenauigkeit des Pirani-Messgerätes, $\SI{20}{\percent}$.
Der Fehler des logarhithmischen Ausdrucks wird nach \eqref{eqn:gaußerr} mit der folgenden Formel bestimmt:
\begin{equation}
  \Delta \ln \Bigl(\frac{p(t)-p_E}{p_0-p_E}\Bigr) = \sqrt{\frac{\Delta_{p(t)}^{2}}{\left(p(t) - p_{E}\right)^{2}} + \frac{\Delta_{p_{0}}^{2}}{\left(p_{0} - p_{E}\right)^{2}} + \frac{\Delta_{p_{E}}^{2} \left(p_{0} - p_{E}\right)^{2}}{\left(p(t) - p_{E}\right)^{2}} \left(\frac{p(t) - p_{E}}{\left(p_{0} - p_{E}\right)^{2}} - \frac{1}{p_{0} - p_{E}}\right)^{2}}.
  \label{eqn:lnperr}
\end{equation}
Die so bestimmten Werte befinden sich ebenfalls in Tabelle \ref{tab:drehevak}.
Nun werden die logarhithmierten Drücke gegen die gemittelten Zeiten aufgetragen und mit der Gleichung $y=ax+b$ in zwei Bereichen optimiert.
Der Plot ist in Abbildung \ref{fig:drehevak} zu sehen. Die Fitparameter wurden zu
\begin{align*}
  a_1 &= \SI{-0.094\pm0.002}{\per\second} \\
  a_2 &= \SI{-0.051\pm0.004}{\per\second} \\
  b_1 &= \SI{-0.169\pm0.107}{} \\
  b_2 &= \SI{-3.648\pm0.397}{}
\end{align*}
bestimmt.
Durch Vergleich mit \eqref{eqn:lnpkurve} lässt sich so das jeweilige Saugvermögen mit
\begin{equation}
  S=-a \cdot V
  \label{eqn:saugvermögen}
\end{equation}
und der zugehörige Fehler mit
\begin{equation}
  \Delta S = \sqrt{V^{2} \Delta_{a}^{2} + a^{2} \Delta_{V}^{2}}
  \label{eqn:saugvermögenerr}
\end{equation}
berechnen.
So wurden die Saugvermögen zu
\begin{align*}
  S_1 &= \SI{1.04\pm0.08}{\litre\per\second} \\
  S_2 &= \SI{0.57\pm0.06}{\litre\per\second}
\end{align*}
bestimmt.
\begin{table}
  \centering
  \caption{Gemessene und berechnete Werte zur Evakuierungsmessung der Drehschieberpumpe.}
  \label{tab:drehevak}
  \begin{tabular}[t]{c c c c c c c c}
   \toprule
    p(t) / $\si{\milli\bar}$ & $\ln \Bigl(\frac{p(t)-p_E}{p_0-p_E}\Bigr)$ & $t_1$ / $\si{\second}$ & $t_2$ / $\si{\second}$ & $t_3$ / $\si{\second}$ & $t_4$ / $\si{\second}$ & $t_5$ / $\si{\second}$ &  $t$ / $\si{\second}$ \\
     \midrule
     \csvreader[no head,
     late after line=\\,
     late after last line=\\\bottomrule]%
     {data/drehevaktab.csv}{}%
     {$\num{\csvcoli}\pm\num{\csvcolii}$ & $\num{\csvcoliii} \pm \num{\csvcoliv}$ & $\num{\csvcolv}$ & $\num{\csvcolvi}$& $\num{\csvcolvii}$& $\num{\csvcolviii}$& $\num{\csvcolix}$& $\num{\csvcolx}\pm\num{\csvcolxi}$ }%
   \end{tabular}
 \end{table}
 \begin{figure}
     \centering
     \includegraphics{build/drehevak.pdf}
     \caption{Plot zur Evakuierungsmessung der Drehschieberpumpe.}
     \label{fig:drehevak}
 \end{figure}
 \FloatBarrier
\subsubsection{Leckratenmessung}
\FloatBarrier
\label{sec:drehleck}
Die Messdaten für die Gleichgewichtsdrücke $p_{g,1}=\SI{0.1}{\milli\bar}$, $p_{g,2}=\SI{0.4}{\milli\bar}$, $p_{g,3}=\SI{0.8}{\milli\bar}$ und $p_{g,4}=\SI{1}{\milli\bar}$ befinden sich in den
Tabellen \ref{tab:drehleck1}, \ref{tab:drehleck2}, \ref{tab:drehleck3} und \ref{tab:drehleck4}. Die Fehler der gemessenen Drücke berechnen sich wie in \ref{sec:drehevak} und die gemessenen
Zeiten werden wieder mit \eqref{eqn:mittelwert} und \eqref{eqn:mittelwerterr} gemittelt.
Die Drücke werden gegen die Zeiten aufgetragen und linear gefittet. Die Plots sind in den Abbildungen \ref{fig:drehleck1}, \ref{fig:drehleck2}, \ref{fig:drehleck3} und \ref{fig:drehleck4} zu sehen.
Die bestimmten Parameter sind
\begin{align*}
  a_1 &= \SI{4.804\pm0.242e-3}{\milli\bar\per\second}\\
  a_2 &= \SI{3.299\pm0.059e-2}{\milli\bar\per\second}\\
  a_3 &= \SI{9.478\pm0.148e-2}{\milli\bar\per\second}\\
  a_4 &= \SI{1.235\pm0.011e-1}{\milli\bar\per\second}\\
  b_1 &= \SI{0.144\pm0.023}{\milli\bar} \\
  b_2 &= \SI{0.345\pm0.030}{\milli\bar} \\
  b_3 &= \SI{0.735\pm0.061}{\milli\bar} \\
  b_1 &= \SI{0.958\pm0.048}{\milli\bar}
\end{align*}
Der Parameter $a$ stellt hierbei die Steigung $\frac{\mathrm{d}p}{\mathrm{d}t}$ dar und so lässt sich das Saugvermögen durch Vergleich mit \eqref{eqn:leck}
folgendermaßen bestimmen:
\begin{equation}
  S=\frac{V}{p_g} \cdot a .
  \label{eqn:leck2}
\end{equation}
Der zugehörige Fehler wird nach \eqref{eqn:gaußerr} mit
\begin{equation}
  \Delta S = \sqrt{\frac{V^{2} a^{2}}{p_{g}^{4}} \Delta_{p_{g}}^{2} + \frac{V^{2} \Delta_{a}^{2}}{p_{g}^{2}} + \frac{a^{2} \Delta_{V}^{2}}{p_{g}^{2}}}.
  \label{eqn:leckerr}
\end{equation}
bestimmt.
So wurden folgende Saugvermögen bestimmt:
\begin{align*}
  S_1 &= \SI{0.53\pm0.05}{\litre\per\second} \\
  S_2 &= \SI{0.92\pm0.07}{\litre\per\second} \\
  S_3 &= \SI{1.32\pm0.10}{\litre\per\second} \\
  S_4 &= \SI{1.37\pm0.10}{\litre\per\second}
\end{align*}
\begin{figure}
    \centering
    \includegraphics{build/drehleck1.pdf}
    \caption{Plot zur Leckratenmessung der Drehschieberpumpe bei $p_g=\SI{0.1}{\milli\bar}$.}
    \label{fig:drehleck1}
\end{figure}
\begin{figure}
    \centering
    \includegraphics{build/drehleck2.pdf}
    \caption{Plot zur Leckratenmessung der Drehschieberpumpe bei $p_g=\SI{0.4}{\milli\bar}$.}
    \label{fig:drehleck2}
\end{figure}
\begin{figure}
    \centering
    \includegraphics{build/drehleck3.pdf}
    \caption{Plot zur Leckratenmessung der Drehschieberpumpe bei $p_g=\SI{0.8}{\milli\bar}$.}
    \label{fig:drehleck3}
\end{figure}
\begin{figure}
    \centering
    \includegraphics{build/drehleck4.pdf}
    \caption{Plot zur Leckratenmessung der Drehschieberpumpe bei $p_g=\SI{1}{\milli\bar}$.}
    \label{fig:drehleck4}
\end{figure}
\begin{table}
  \centering
  \caption{Gemessene und berechnete Werte zur Leckratenmessung der Drehschieberpumpe bei $p_g=\SI{0.1}{\milli\bar}$.}
  \label{tab:drehleck1}
  \begin{tabular}[t]{c c c c c}
   \toprule
    p(t) / $\si{\milli\bar}$ & $t_1$ / $\si{\second}$ & $t_2$ / $\si{\second}$ & $t_3$ / $\si{\second}$ & $t$ / $\si{\second}$ \\
     \midrule
     \csvreader[no head,
     late after line=\\,
     late after last line=\\\bottomrule]%
     {data/drehleck1tab.csv}{}%
     {$\num{\csvcoli}\pm\num{\csvcolii}$ & $\num{\csvcoliii}$ & $\num{\csvcoliv}$ & $\num{\csvcolv}$ & $\num{\csvcolvi}\pm\num{\csvcolvii}$ }%
   \end{tabular}
 \end{table}
 \begin{table}
   \centering
   \caption{Gemessene und berechnete Werte zur Leckratenmessung der Drehschieberpumpe bei $p_g=\SI{0.4}{\milli\bar}$.}
   \label{tab:drehleck2}
   \begin{tabular}[t]{c c c c c}
    \toprule
     p(t) / $\si{\milli\bar}$ & $t_1$ / $\si{\second}$ & $t_2$ / $\si{\second}$ & $t_3$ / $\si{\second}$ & $t$ / $\si{\second}$ \\
      \midrule
      \csvreader[no head,
      late after line=\\,
      late after last line=\\\bottomrule]%
      {data/drehleck2tab.csv}{}%
      {$\num{\csvcoli}\pm\num{\csvcolii}$ & $\num{\csvcoliii}$ & $\num{\csvcoliv}$ & $\num{\csvcolv}$ & $\num{\csvcolvi}\pm\num{\csvcolvii}$ }%
    \end{tabular}
  \end{table}
  \begin{table}
    \centering
    \caption{Gemessene und berechnete Werte zur Leckratenmessung der Drehschieberpumpe bei $p_g=\SI{0.8}{\milli\bar}$.}
    \label{tab:drehleck3}
    \begin{tabular}[t]{c c c c c}
     \toprule
      p(t) / $\si{\milli\bar}$ & $t_1$ / $\si{\second}$ & $t_2$ / $\si{\second}$ & $t_3$ / $\si{\second}$ & $t$ / $\si{\second}$ \\
       \midrule
       \csvreader[no head,
       late after line=\\,
       late after last line=\\\bottomrule]%
       {data/drehleck3tab.csv}{}%
       {$\num{\csvcoli}\pm\num{\csvcolii}$ & $\num{\csvcoliii}$ & $\num{\csvcoliv}$ & $\num{\csvcolv}$ & $\num{\csvcolvi}\pm\num{\csvcolvii}$ }%
     \end{tabular}
   \end{table}
   \begin{table}
     \centering
     \caption{Gemessene und berechnete Werte zur Leckratenmessung der Drehschieberpumpe bei $p_g=\SI{1}{\milli\bar}$.}
     \label{tab:drehleck4}
     \begin{tabular}[t]{c c c c c}
      \toprule
       p(t) / $\si{\milli\bar}$ & $t_1$ / $\si{\second}$ & $t_2$ / $\si{\second}$ & $t_3$ / $\si{\second}$ & $t$ / $\si{\second}$ \\
        \midrule
        \csvreader[no head,
        late after line=\\,
        late after last line=\\\bottomrule]%
        {data/drehleck4tab.csv}{}%
        {$\num{\csvcoli}\pm\num{\csvcolii}$ & $\num{\csvcoliii}$ & $\num{\csvcoliv}$ & $\num{\csvcolv}$ & $\num{\csvcolvi}\pm\num{\csvcolvii}$ }%
      \end{tabular}
    \end{table}
    \FloatBarrier
\subsection{Turbopumpe}
