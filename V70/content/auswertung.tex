\section{Auswertung}
\label{sec:Auswertung}
\subsection{Volumenbestimmung}
\begin{figure}
  \centering
  \includegraphics{images/pumpstand3.jpg}
  \caption{Der verwendete Pumpstand mit nummerierten Bauteilen.}
  \label{fig:ps1}
\end{figure}
\begin{figure}
  \centering
  \includegraphics{images/pumpstand5.jpg}
  \caption{Der verwendete Pumpstand mit nummerierten Bauteilen.}
  \label{fig:ps2}
\end{figure}
\begin{figure}
  \centering
  \includegraphics{images/pumpstand4.jpg}
  \caption{Der verwendete Pumpstand mit nummerierten Bauteilen.}
  \label{fig:ps3}
\end{figure}
Die Volumina der benutzten Bauteile sind in Tabelle \ref{tab:volumina} aufgelistet.
Die Volumina 2 und 8 wurden dabei selber gemessen, die restlichen Werte wurden aus \cite{leitfaden} entnommen. Zur Bestimmung des Volumens von Bauteil 2 wurden folgende Werte gemessen:
\begin{align*}
  d_\text{innen}&= \SI{40\pm0.5}{\milli\metre} \\
  l_1 &= \SI{128.2\pm0.5}{\milli\metre}\\
  l_2 &= \SI{45\pm2}{\milli\metre}
\end{align*}
Das Volumen berechnet sich dann mit der Formel
\begin{equation*}
  V_2 = \pi (\frac{d_\text{innen}}{2})^2 \cdot (l_1 + l_2).
\end{equation*}
zu dem in \ref{tab:volumina} angegebenen Wert. Bei Bauteil 8 wurden folgende Werte und, da es sich um ein Kreuzstück handelt, Formel genutzt:
\begin{align*}
  d_\text{innen}&= \SI{10\pm0.5}{\milli\metre} \\
  l_1 &= \SI{80\pm0.5}{\milli\metre}\\
  l_2 &= \SI{30\pm1}{\milli\metre} \\
  V_8 &= \pi (\frac{d_\text{innen}}{2})^2 \cdot (l_1 + 2 l_2).
\end{align*}
Bei den Messungen mit der Drehschieberpumpe ist das Ventil 6, sowie das in \ref{fig:ps2} zu sehende Ventil 4 verschlossen.
Die Addition der verbleibenden Volumina ergibt ein Gesamtvolumen von $\SI{11.1\pm0.8}{\litre}$.
Bei den Messungen mit der Trubopumpe sind diese Ventile offen, wodurch sich ein Gesamtvolumen von $\SI{11.2\pm0.8}{\litre}$ ergibt. 
\subsection{Drehschieberpumpe}
\subsubsection{Evakuierungsmessung}
