\section{Durchführung}
\label{sec:Durchführung}
Zuerst wird der Pumpstand wie in den Abbildungen \ref{fig:ps1} bis \ref{fig:ps3} aufgebaut. Die Zahlen sind dabei die einzelnen verwendeten Bauteile. 1 ist der Rezipient in dem das Vakuum
erzeugt werden soll, welcher mit 2 und 3 als Rohrleitungen verbunden wird. An dem T-Stück 2 wird eine Kaltkathoden-Ionisationskammer angeschlossen. An dem großen Kreuzstück 3 wird an einer
Seite ein Kugelventil 4 und ein Dosierventiel 11 angeschlossen und auf der gegenüberliegende Ende ein Kugelventil, welches mit einem kurzen Schlauch 5 mit dem kleinen Kreuzstück 8 verbunden ist.
Das letzte Ende des Kreuzstücks 3 wird mit einem T-Stück verbunden, an dessen einem Ende eine Glühkathoden-Ionisationskammer angeschlossen wird und an der anderen Seite ein Klappenventil 6,
welches mit der Turbomolekularpumpe 7 verbunden ist. Diese ist über ein Kugelventil ebenfalls mit dem Kreuzstück 8 verbunden. Am dritten Verbindungsstück des Kreuzstückes ist ein
Vakuummeter nach Pirani angebaut und am letzten Ende ein Schlauch 9, welcher mit der Drehschieberpumpe 10 verbunden ist.
\begin{figure}
  \centering
  \includegraphics[scale=0.4]{images/pumpstand3.jpg}
  \caption{Der verwendete Pumpstand mit nummerierten Bauteilen.}
  \label{fig:ps1}
\end{figure}
\begin{figure}
  \centering
  \includegraphics[scale=0.4]{images/pumpstand5.jpg}
  \caption{Der verwendete Pumpstand mit nummerierten Bauteilen.}
  \label{fig:ps2}
\end{figure}
\begin{figure}
  \centering
  \includegraphics[scale=0.4]{images/pumpstand4.jpg}
  \caption{Der verwendete Pumpstand mit nummerierten Bauteilen.}
  \label{fig:ps3}
\end{figure}\FloatBarrier
Anschließend wird durch das Einschalten der Drehschieberpumpe der Aufbau auf seine Dichtigkeit überprüft. Es sollte sich ein Enddruck im Bereich von $\SI{0,03}{\milli\bar}$ bis $\SI{0,05}{\milli\bar}$
einstellen. Als Messgerät wird das Pirani-Vakuummeter verwendet. Wenn dies nicht erreicht wird, sollten die Ventile und Abdichtungen zwischen zwei Bauteilen überprüft werden.
Desweiteren wird der Rezipient einige Minuten mit der Turbopumpe evakuiert, um mögliche Wasseranlagerungen von der Innenoberfläche zu entfernen und damit die Desorptionsrate 
zu vermindern.

\subsection{Messungen zur Drehschieberpumpe}
Zur Aufnahme der Evakuierungskurve der Drehschieberpumpe wird zuerst die Turbopumpe abgeklemmt. Dazu wird das Kugelventil am Fuße der Turbopumpe und das Klappenventil 6 geschlossen. Dann
wird mit der Drehschieberpumpe der Rezipient evakuiert bis ein Enddruck $p_{\mathrm{E}}$ erreicht ist, welcher notiert wird. Mittels des Dosierventils 11 wird der Rezipient
belüftet, bis der Normaldruck erreicht ist. Der Rezipient wird nun 5 mal evakuiert und dazwischen wieder belüftet. Es werden bei vorher festgelegten Drücken die benötigten Zeiten mittels der Stoppuhr
eines Handys gemessen.\\
Für die Leckratenmessung zur Bestimmung des Saugvermögens werden bei laufender Pumpe mittels des Dosierventils Gleichgewichtsdrücke $p_{\mathrm{g}}$ eingestellt. Diese Betragen
$\SI{0,1}{\milli\bar}$, $\SI{0,4}{\milli\bar}$, $\SI{0,8}{\milli\bar}$ und $\SI{1,0}{\milli\bar}$. Wenn diese eingestellt sind, wird die Pumpe abgeklemmt und über ca. eine Größenordnung
der zunehmende Druck gemessen. Es wird wieder für vorher festgelegte Drücke die Zeit gemessen und notiert. Für jeden Gleichgewichtsdruck wird die Messung drei mal wiederholt.

\subsection{Messungen zur Turbomolekularpumpe}
Für die Messung mit der Turbopumpe werden die beiden Ventile an der Pumpe wieder geöffnet und das Ventil am Schlauch 5 wird geschlossen. Die Turopumpe wird erst eingeschaltet, wenn das
Vorvakuum durch die Drehschieberpumpe besser als $\SI{0,1}{\milli\bar}$ ist. Nach einiger Zeit stellt sich wieder ein Enddruck $p_{\mathrm{E}}$ ein, welcher wieder notiert wird.
Die Messung der Evakuierungskurve wird wie zuvor durchgeführt, nur das der Rezipient mit dem Dosierventil in diesen Teil auf einen Druck von $\SI{5e-3}{\milli\bar}$ eingestellt wird,
bevor der Rezipient wieder evakuiert wird. Desweiteren wird als Messgerät diesmal die Glühkathoden-Ionisationskammer benutzt. Damit diese nicht durchbrennt, sollte sie nur benutzt werden
wenn der Druck geringer als $\SI{e-2}{\milli\bar}$ ist.\\
Die Leckratenmessung läuft ebenfalls analog ab wie zuvor. Die einzustellenen Gleichgewichtsdrücke betragen $\SI{5e-5}{\milli\bar}$, $\SI{10e-5}{\milli\bar}$, $\SI{15e-5}{\milli\bar}$ und
$\SI{20e-5}{\milli\bar}$ und die Messung wird wieder für jeden eingestellten Druck drei mal durchgeführt.

\subsection{Volumenbestimmung}
Zuletzt werden von dem großen T-Stück 2 und dem kleinen Kreuzstück 8 die Volumina mittels Lineal und Schieblehre bestimmt. Die Fehler für die einzelnen Längen werden abgeschätzt.
Die Volumina der restlichen Bauteile werden aus dem Leitfaden\cite{leitfaden} entnommen.
