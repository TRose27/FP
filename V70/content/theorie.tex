\section{Theorie}
\label{sec:Theorie}
Als Vakuum wird der Zustand eines Gases bezeichnet, dessen Druck in einem Behälter kleiner ist als der Umgebungsdruck.\cite{pfeiffer} Dies bedeutet auf der Erde, dass der Druck in einem
Behälter kleiner, als der Atmosphärendruck $p_0 = \SI{1013}{\milli\bar}$ sein muss. Die sich im Behälter befindene Luft wurde in diesem Versuch als ideales Gas angenommen. Es wird daher davon
ausgegangen das die Teilchen nur elastische Stöße untereinander und mit den Wänden des Behälters ausführen. Jegliche andere Wechselwirkung wie z.B. zwischen den einzelnen Teilchen werden
als vernachlässigbar betrachtet. Die Gasteilchen werden außerdem als ausdehnungslose Massepunkte betrachtet. Die allgemeine Gasgleichung für ideale Gase lautet
\begin{equation}%ideale Gasgleichung
    p \cdot V = n \cdot R \cdot T = N \cdot k_{\mathrm{B}} \cdot T,
    \label{eqn:idgas}
\end{equation}
wobei $n$ für die Stoffmenge, $N$ für die Anzahl der Teilchen und $R$ für die allgemeine Gaskonstante steht. Für eine konstante Temperatur $T$ und eine konstante Stoffmenge $n$ folgt
das der Druck $p$ antiproportional zum Volumen $V$ ist. Dieser Zusammenhang wird durch das Boyle'sches Gesetz beschrieben:
\begin{equation}%Boyle'sches Gesetz
    p \propto \frac{1}{V} \;.
    \label{eqn:boyle}
\end{equation}
Der Druck ist dabei die Kraft pro Fläche $p = \dfrac{F}{A}$ und hat die Einheit $[P] = \SI{1}{bar} = \SI{e5}{\pascal}$. Im folgenden werden alle Drücke immer in Bar angegeben.\\
Die Partialdrücke, das heißt die Teildrücke der unterschiedlichen Fraktionen, werden ebenfalls vernachlässig, da angenommen wird das die Luft im Behälter nur aus einer Fraktion besteht.

\subsection{Herleitung der p(t)-Kurve}
Unter den weiteren Annahmen das das Volumen beim messen des Druckes konstant bleibt und das der Druck $p(t)$ zu einem Zeitpunkt immer im gesamten Volumen herscht, sowie es keine Lecks und
Ausgasungen von den Oberflächen gibt und das Saugvermögen $S$ konstant und unabhängig vom Druck ist, lässt sich der Druckverlauf aus der idealen Gasgleichung herleiten.\cite[12]{leitfaden}\\
Unter Berücksichtigung der getroffenen Annahmen folgt aus der zeitlichen Ableitung der Gleichung \eqref{eqn:idgas}
\begin{align}
    \frac{\mathrm{d}}{\mathrm{d}t}pV = \frac{\mathrm{d}}{\mathrm{d}t}nRT \nonumber\\
    \implies \dot{p}V + p\dot{V} = 0,\label{eqn:idgasableitung}
\end{align}
da die Stoffmenge, Temperatur und Gaskonstante konstant sind. Da das Saugvermögen $S$ konstant sein soll, ist es gleich der Zunahme des Volumens pro Zeit:
\begin{equation*}
    \dot{V} = S.
\end{equation*}
Damit folgt aus Gleichung \eqref{eqn:idgasableitung}:
\begin{align}
    \dot{p}V = -pS. \label{eqn:pVpS}
\end{align}
Der Lösungsansatz dieser bekannten Differenzialgleichung lautet\cite[12]{leitfaden}:
\begin{align*}
    p(t) &= p_0 \exp\left(-\frac{t}{\tau}\right),\\
    \dot{p}(t) &= -\frac{p_0}{\tau}\exp\left(-\frac{t}{\tau}\right) = -\frac{p(t)}{\tau},
\end{align*}
dabei ist $p_0$ der Anfangsdruck und $\tau$ eine zu bestimmenden Zeitkonstante. Durch einsetzen in die Gleichung \eqref{eqn:pVpS} ergibt sich
\begin{equation*}
    \tau = \frac{V}{S}.
\end{equation*}
Unter Berücksichtigung eines endlichen Enddrucks $p_{\mathrm{E}}$ ergibt sich durch das einsätzen in den Lösungsansatz die Funtkon $p(t)$ in Abhängigkeit vom Saugvermögen $S$:
\begin{equation}
    p(t) = (p_0 - p_{\mathrm{E}})\exp\left(-t\frac{S}{V}\right) + p_{\mathrm{E}}.
    \label{eqn:pkurve}
\end{equation}
Der Enddruck $p_{\mathrm{E}}$ ergibt sich aus nicht vermeidbaren Leckagen und Ausgasungen\cite[13]{leitfaden}. Der schematische Kurvenverlauf ist in Abbildung \ref{fig:pkurve} zu sehen.
\begin{figure}[H]
    \centering
    \includegraphics[scale=0.7]{images/p-kurve.png}
    \caption{Schematischer Kurvenverlauf der p(t)-Kurve.\cite[13]{leitfaden}}
    \label{fig:pkurve}
\end{figure}\FloatBarrier

\subsection{Leckratenmessung}
Mit der als $Q= V \cdot \frac{\mathrm{d}p}{\mathrm{d}t}$ definierten Leckrate, lässt sich das Saugvermögen, nach Einstellung eines Gleichgewichtsdruckes $p_g$, als
\begin{equation}
  S = \frac{Q}{p_g} = \frac{V}{p_g} \cdot \frac{\mathrm{d}p}{\mathrm{d}t}
  \label{eqn:leck}
\end{equation}
ausdrücken.\cite{leitfaden}

\subsection{Druckbereiche}
Ein Vakuum lässt sich je nachdem wie viele Teilchen noch im Volumen vorhanden sind, in unterschiedliche Druckbereiche einteilen. Diese sind in Abbildung \ref{fig:druckbereiche} zu sehen.
Je niedriger der Druck ist, desto größer ist die Qualität des Vakuums.
\begin{figure}%Druckbereiche
    \centering
    \includegraphics[width=\textwidth]{images/druckbereiche.jpg}
    \caption{Druckbereiche in der Vakuumtechnik.\cite{bereiche}}
    \label{fig:druckbereiche}
\end{figure}\FloatBarrier
Die Anzahl an Teilchen im Volumen geht mit der mittleren freien Weglänge der Teilchen einher. Diese steht für die gemittelte Distanz die ein Teilchen zurücklegen muss, um mit einem anderen
Teilchen zu stoßen. Sie wird mit abnehmener Teilchenzahl im Volumen größer, da weniger Teilchen zum kollidieren vorhanden sind.

\subsection{Strömungsarten und Leitwert}
Aus dem Verhlältnis der mittleren freien Weglänge und dem Durchmesser eines Strömungskanals lassen sich unterschiedliche Strömungsarten beschreiben.  Das Verhältnis wird durch die Knudsenzahl
\begin{equation*}
    \text{Kn} = \frac{\lambda}{d}
\end{equation*}
beschrieben, wobei $\lambda$ für die mittlere freie Weglänge und $d$ für den Durchemesser steht. Je nachdem welchen Wert die Knudsenzahl hat, liegt eine andere Art von Strömung vor, welche
in  unterschiedlichen Druckbereichen vorliegen. Bei kleinen Knudsenzahlen ist der Durchmesser des Strömungskanals deutlich größer als die freie mittlere Weglänge. Daher stoßen in diesem
Bereich die Gasteilchen sehr oft untereinander und nur selten mit den Wänden. Diese Strömungsart wird viskose Strömung genannt und lässt sich in laminare und turbulente Strömungen unterteilen.
Bei der laminaren Strömung bleiben die Teilchen in parallelen Schichten zueinander und es gibt daher keine Wirbel. Wenn die kritische Reynolds-Zahl
\begin{equation*}
    \text{Re} = \frac{\rho v l}{\eta},
\end{equation*}
wobei $\rho$ die Dichte, $v$ die Geschwindigkeit, $l$ die charakteristische Länge und $\eta$ die Viskosität ist, überschritten wird, wird der Bereich der turbulenten Strömung erreicht.
Hier strömen die Teilchen völlig ungeordnet durcheinander. Dies wird in der Vakuumtechnik versucht zu vermeiden, da aufgrund der großen Strömungswiderständen sonst ein höheres Saugvermögen
vonnöten wär.\cite{strömung}\\
Wenn die mittlere Weglänge viel größer ist als der Durchmesser liegt die molekulare Strömung vor. Daher kommt es kaum noch zu stößen unter den Gasteilchen, sondern fast nur noch mit den
Wänden des Behälters.\\
Durch Reibung der Gasteilchen mit den Wänden oder untereinander kommt es zu Strömungswiderständen $W$ in den Rohrleitungen mit den eine Vakuumpumpe und der Rezipient verbunden sind. Die Widerstände
drücken sich in Druckunterschiede und Saugvermögenverlust aus. Der Leitwert ist dabei der Kehrwert der Strömungswiderstände und wird allgemein in der Vakuumtechnik verwendet. Die Formel des
Leitwerts $C$ lautet
\begin{equation}
    C = \frac{l}{W} = \frac{q_p v}{\Delta p},
    \label{eqn:leitwert}
\end{equation}
wobei $l$ die Länge der verwendeten Rohrleitungen, $q_p v$ die pro Zeiteinhat durchströmende Gasmenge und $\Delta p$ der anliegende Druckgradient ist.
Bei molekularen Strömungen gilt
\begin{equation}
  q_p v = A \cdot \frac{\overline{c}}{4} \cdot (p1-p2)
  \label{eqn:leitwertmolekular}
\end{equation}
mit dem Querschnitt $A$ und der mittleren thermischen Geschwindigkeit $\overline{c}$.\cite{leitwert}
Da der Leitwert, vorallem bei molekularen Strömungen, einen großen Einfluss auf das Saugvermögen hat, wird das effektive Saugvermögen folgendermaßen definiert \cite{leitfaden}:
\begin{equation}
  \frac{1}{S_\text{eff}} = \frac{1}{S_0}+\frac{1}{C}
  \label{eqn:seff}
\end{equation}

\subsection{Sorption}
Sorption ist ein Überbegriff für das anreichern eines Stoffes innerhalb einer Phase oder an der Grenzfläche zwischen zwei Phasen. Bei der Absorption absorbiert eine Phase z.B. ein Feststoff
Teilchen aus einer anderen Phase z.B ein Gas. Die Gasteilchen werden dabei in das freie Volumen des Feststoffes absorbiert. Bei der Adsorption lagern sich die Teilchen bei diesem Beispiel
dahingegen an der Grenzfläche des Feststoffes ein und dringen nicht in das Volumen des Feststoffes ein.
Der Umkehrprozess zu den beiden Vorgängen lauter Desorption. Dieser Effekt ist der Grund für die Existenz von "virtuellen" Lecks. Diese sind nicht detektierbar und können daher auch nicht
wie normale Lecks verhindert werden. Da Desorption durch das Material beeinflusst wird, ist es nötig dies schon bei der Konstruktion einer Vakuumpumpe zu berücksichtigen.

\subsection{Vakuumerzeugung}
Um ein Vakuum zu erzeugen, müssen aus einem zuvor gasgefüllten Volumen, Gasteilchen entfernt werden. Dies wird durch Vakuumpumpen realisiert. Diese werden durch ihr Wirkungsprinzip in
Gastransfer- und gasbindene Vakuumpumpen unterschieden. Gastransferpumpen sind entweder Verdrängerpumpen, welche die Teilchen in einem geschlossenen Arbeitsraum, oder durch Impulsübertragung
transportieren. Bei Verdrängerpumpen wird durch das sich änderne Volumen im Arbeitsraum das gesamt Volumen verändert. Nach dem Boyle'schen Gesetz \eqref{eqn:boyle} wird daher der Druck,
bei konstarnter Temperatur, im Rezipienten kleiner. Der Arbeitsraum wird daraufhin abgeschlossen und das Gas nach eventueller verdichtung ausgestoßen.\\
Gasbindene Vakuumpumpen arbeiten indem sie die Gasteilchen an Festkörperoberflächen mittels absorption und adsoprtion binden und dadurch den Druck im Rezipienten verringern.
\subsubsection{Drehschieberpumpe}
\begin{figure}[H]
    \centering
    \includegraphics[width=\textwidth]{images/drehpumpe.png}
    \caption{Schematischer Aufbau einer Drehschieberpumpe.\cite{drehpumpe}}
    \label{fig:drehpumpe}
\end{figure}\FloatBarrier
In Abbildung \ref{fig:drehpumpe} ist der schematische Aufbau einer Drehschieberpumpe dargestellt. Der exzentrisch eingebaute Rotor trennt den Einlass vom Auslass. Das Gehäuse
(Stator), der Rotor und die Schieber begrenzen den Arbeitsraum. Die Schieber sind beweglich und werden beim drehen des Rotors aus diesem rausgeschoben um den Arbeitsraum in zwei Teile zu
trennen. Wenn der Rotor sich dreht vergrößert sich das Volumen im Arbeitsraum bis der zweite Schieber den Arbeitsraum abtrennt. Wie bereits erwähnt wird durch diese Volumenvergrößerung
der Druck im Rezipienten kleiner. Das eingeschlossene Gas wird noch komprimiert bis der Druck groß genug ist um das Auslassventil gegen den Atmosphärendruck zu öffnen. Dadurch gelangt Öl
vom ölüberlagerten Auslassventil in den Stator, wodurch die Schieber geölt werden und gegen den Stator abgedichtet werden. Der Arbeitsbereich einer Drehschieberpumpe liegt im Grob- und
Feinvakuum, weshalb sie auch gerne als Vorpumpe verwendet wird.\cite{drehpumpe}
\subsubsection{Turbomolekularpumpe}
\begin{figure}[H]
    \centering
    \includegraphics[width=\textwidth]{images/turbopumpe.png}
    \caption{Schematischer Aufbau einer Turbomolekularpumpe.\cite{turbopumpe}}
    \label{fig:turbopumpe}
\end{figure}\FloatBarrier
Eine Turbomolekularpumpe oder auch einfach Turbopumpe ist, wie in Abbildung \ref{fig:turbopumpe} zu sehen, aufgebaut.  Diese Pumpenart basiert auf der Impulsübertragung von den rotierenden
Rotorblätter auf die Gasteilchen. Wenn die Rotorblätter ein Gasteilchen treffen binden sich dieses durch adsorption für eine gewisse Zeit an das Rotorblatt. Bevor das Teilchen wieder desorbiert,
addiert sich die Rotorgeschwindigkeit und es übernimmt die Impulsrichtung. Deswegen sind die Rotorblätter auch um den Winkel $\alpha$ zur horizontalen gedreht. Damit die Teilchen nicht die
von den Rotorblättern übertragene Geschwindigkeitkomponente durch Stöße mit anderen Teilchen verlieren, ist eine molekulare Strömung vonnöten. Deswegen funktioniert eine Turbopumpe auch nur
mit einer Vorpumpe, welche in diesem Versuch durch eine Drehschieberpumpe realisiert wird. Der Arbeitsbereich einer Turbopumpe reicht vom Feinvakuum bis hin zum Ultrahochvakuum.\cite{turbopumpe}

\subsection{Vakuummessung}
Zur Druckmessung im Hoch- und Ultrahochvakuumbereich werden Ionisations-Vakuummeter verwendet, welche auf den Printzip der indirekten Druckmessung mittels elektrischer Größen basiert. Dazu
muss das Restgas ioniesiert werden. Die in diesem Versuch verwendeten Kaltkathoden- und Glühkathoden-Ionisationskammern sind zwei verschiedene Varianten des Ionisations-Vakuummeters.
Bei der Glühkathode werden vorhandene Elektronen zwischen einer Kathode und Anode, an denen eine Gleichspannung von ca. $\SI{2}{\kilo\volt}$ anliegt, beschleunigt und können so die restlichen
Gasteilchen ionisieren. Dieser Entladungsstrom ist messbar und proportional zum Druck. Das Glühkathoden-Vakuummeter beschleunigt keine vorhandene Elektronen, sondern erzeugt diese selber
mit einer Glühkathode. Diese werden dann anschließend beschleunigt und ioniesieren die Gasteilchen, wodurch ein Strom messbar ist, welcher von der Teilchenzahldichte im Restgas abhängt.
In den Druckbereichen unterm Hochvakuumbereich wurde in diesem Versuch das Wärmeleitungs-Vakuummeter nach Pirani verwendet. Dieses nutzt das in gewissen Grenzen die Wärmeleitfähigkeit von
Gasen druckabhängig ist. In Abbildung \ref{fig:pirani} ist der lineare Bereich der Druckbereich, in den das Vakuummeter eingesetzt wird.
\begin{figure}[H]
    \centering
    \includegraphics{images/pirani.jpg}
    \caption{Schematische druckabhängige Wärmeabgabe.\cite{pirani}}
    \label{fig:pirani}
\end{figure}\FloatBarrier

\subsection{Anwendugsfelder der Vakuumphysik}
Ein einfaches Beispiel für die nutzung von Vakua ist zur konservierung von verderblichen Produkten, da bei geringer Luftsauerstoffmenge Verwesungsprozesse nur stark eingeschränk statt finden.
Desweiteren werden Vakua zur wärmebehandlung von Metallen verwendet, da so das Oxidieren durch Sauerstoff verhindert werden kann. Ein Grobvakuum wird in der Industrie auch als Sauggreifer
verwendet um flächige Werkstücke zu greifen und zu transportieren.\cite{anwendung}
