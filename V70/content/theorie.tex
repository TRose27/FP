\section{Theorie}
\label{sec:Theorie}
Als Vakuum wird der Zustand eines Gases bezeichnet, dessen Druck in einem Behälter kleiner ist als der Umgebungsdruck.\cite{pfeiffer} Dies bedeutet auf der Erde, dass der Druck in einem
Behälter kleiner als der Atmosphärendruck $p_0 = \SI{1013}{\milli\bar}$ sein muss. Die sich im Behälter befindene Luft wurde in diesem Versuch als ideales Gas angenommen. Es wird daher davon
ausgegangen das die Teilchen nur elastische Stöße untereinander und mit den Wänden des Behälters ausführen. Jegliche andere Wechselwirkung wie z.B. zwischen den einzelnen Teilchen werden 
als vernachlässigbar betrachtet. Die Gasteilchen werden außerdem als ausdehnungslose Massepunkte betrachtet. Die allgemeine Gasgleichung für ideale Gase lautet
\begin{equation}%ideale Gasgleichung
    p \cdot V = n \cdot R \cdot T = N \cdot k_{\mathrm{B}} \cdot T,
    \label{eqn:idgas}
\end{equation}
wobei $n$ für die Stoffmenge, $N$ für die Anzahl der Teilchen und $R$ für die allgemeine Gaskonstante steht. Für eine konstante Temperatur $T$ und eine konstante Stoffmenge $n$ folgt 
das der Druck $p$ antiproportional zum Volumen $V$ ist. Dieser Zusammenhang wird durch das Boyle'sches Gesetz beschrieben:
\begin{equation*}%Boyle'sches Gesetz
    p \propto \frac{1}{V} \;.
\end{equation*}
Der Druck ist dabei die Kraft pro Fläche $p = \dfrac{F}{A}$ und hat die Einheit $[P] = \SI{1}{bar} = \SI{e5}{\pascal}$. Im folgenden werden alle Drücke immer in Bar angegeben.\\
Die Partialdrücke, das heißt die Teildrücke der unterschiedlichen Fraktionen, werden ebenfalls vernachlässig, da angenommen wird das die Luft im Behälter nur aus einer Fraktion besteht.

\subsection{Herleitung der p(t)-Kurve}
Unter den weiteren Annahmen das das Volumen beim messen des Druckes konstant bleibt und das der Druck $p(t)$ zu einem Zeitpunkt immer im Gesamten Volumen herscht, sowie es keine Lecks und 
Ausgasungen von den Oberflächen gibt und das Saugvermögen $S$ konstant und unabhängig vom Druck ist, lässt sich der Druckverlauf aus der idealen Gasgleichung.\cite[12]{leitfaden}\\
Unter Berücksichtigung der getroffenen Annahmen folgt aus der zeitlichen Ableitung der Gleichung \eqref{eqn:idgas}
\begin{align}
    \frac{\mathrm{d}}{\mathrm{d}t}pV = \frac{\mathrm{d}}{\mathrm{d}t}nRT \nonumber\\
    \implies \dot{p}V + p\dot{V} = 0,\label{eqn:idgasableitung}
\end{align}
da die Stoffmenge, Temperatur und Gaskonstante konstant sind. Da das Saugvermögen $S$ konstant sein soll, ist es gleich der Zunahme des Volumens pro Zeit:
\begin{equation*}
    \dot{V} = S. 
\end{equation*}
Damit folgt aus Gleichung \eqref{eqn:idgasableitung}:
\begin{align}
    \dot{p}V = -pS. \label{eqn:pVpS} 
\end{align}
Der Lösungsansazu dieser bekannten Differenzialgleichung lautet\cite[12]{leitfaden}: 
\begin{align*}
    p(t) &= p_0 \exp\left(-\frac{t}{\tau}\right),\\
    \dot{p}(t) &= -\frac{p_0}{\tau}\exp\left(-\frac{t}{\tau}\right) = -\frac{p(t)}{\tau},
\end{align*}
dabei ist $p_0$ der Anfangsdruck und $\tau$ eine zu bestimmenden Zeitkonstante. Durch einsetzen in die Gleichung \eqref{eqn:pVpS} ergibt sich 
\begin{equation*}
    \tau = \frac{V}{S}.
\end{equation*}
Unter Berücksichtigung eines endlichen Enddrucks $p_{\mathrm{E}}$ ergibt sich durch das einsätzen in den Lösungsansatz die Funtkon $p(t)$ in Abhängigkeit vom Saugvermögen $S$:
\begin{equation}
    p(t) = (p_0 - p_{\mathrm{E}})\exp\left(-t\frac{S}{V}\right) + p_{\mathrm{E}}.
\end{equation}
Der Enddruck $p_{\mathrm{E}}$ ergibt sich aus nicht vermeidbaren Leckagen und Ausgasungen\cite[13]{leitfaden}. Der schematische Kurvenverlauf ist in Abbildung \ref{fig:pkurve} zu sehen.
\begin{figure}[H]
    \centering 
    \includegraphics[scale=0.7]{p-kurve.png}
    \caption{Schematischer Kurvenverlauf der p(t)-Kurve.\cite[13]{leitfaden}}
    \label{fig:pkurve}
\end{figure}\FloatBarrier

\subsection{Druckbereiche}
Ein Vakuum lässt sich je nachdem wie viele Teilchen noch im Volumen vorhanden sind, in unterschiedliche Druckbereiche einteilen. Diese sind in Abbildung \ref{fig:druckbereiche} zu sehen.
Je niedriger der Druck ist, desto größer ist die Qualität des Vakuums.
\begin{figure}%Druckbereiche
    \centering
    \includegraphics[scale=2.0]{druckbereiche.jpg}
    \caption{Druckbereiche in der Vakuumtechnik.\cite{bereiche}}
    \label{fig:druckbereiche}
\end{figure}\FloatBarrier
Die Anzahl an Teilchen im Volumen geht mit der mittleren freien Weglänge der Teilchen einher. Diese steht für die gemittelte Distanz die ein Teilchen zurücklegen muss um mit einem anderen 
Teilchen zu stoßen. Sie wird mit abnehmener Teilchenzahl im Volumen größer, da weniger Teilchen zum kollidieren vorhanden sind. 