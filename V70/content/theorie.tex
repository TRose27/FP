\section{Theorie}
\label{sec:Theorie}
Als Vakuum wird der Zustand eines Gases bezeichnet, dessen Druck in einem Behälter kleiner ist als der Umgebungsdruck. Dies bedeutet auf der Erde, dass der Druck in einem
Behälter kleiner als der Atmosphärendruck $p_0 = \SI{1013}{\milli\bar}$ sein muss. Der Druck ist dabei die Kraft pro Fläche $P = \dfrac{F}{A}$ und hat die Einheit 
$[P] = \SI{1}{bar} = \SI{e5}{\pascal}$. Im folgenden werden alle Drücke immer in Bar angegeben.\\
Ein Vakuum lässt sich je nachdem wie viele Teilchen noch im Volumen vorhanden sind, in unterschiedliche Druckbereiche einteilen. Diese sind in Abbildung \ref{fig:druckbereiche} zu sehen.
\begin{figure}
    \centering
    \includegraphics{Druckbereiche.png}
    \caption{Druckbereiche in der Vakuumtechnik.}
    \label{fig:druckbereiche}
\end{figure}