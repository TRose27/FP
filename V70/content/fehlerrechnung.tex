\section{Fehlerrechnung}
\label{sec:Fehlerrechnung}
Im Folgenden werden alle Mittelwerte mit folgender Formel bestimmt:
\begin{equation}
  \overline{x}=\frac{1}{N}\sum_{i=1}^N x_i.
  \label{eqn:mittelwert}
\end{equation}
Der zugehörige Fehler des Mittelwertes berechnet sich nach
\begin{equation}
  \Delta\overline{x}=\frac{1}{\sqrt{N}}\sqrt{\frac{1}{1-N}\sum_{i=1}^N (x_i - \overline{x})^2}.
  \label{eqn:mittelwerterr}
\end{equation}
Werden fehlerbehaftete Größen in einer späteren Formel benutzt, so wird der neue Fehler mit Hilfe der Gauß'schen Fehlerfortpflanzung angegeben:
\begin{equation}
  \Delta f=\sqrt{\sum_{i=1}^N \left(\frac{\partial f}{\partial x_i}\right)^2 \cdot (\Delta x_i)^2}.
  \label{eqn:gaußerr}
\end{equation}
Die Regression sowohl von Ausgleichsgeraden als auch von anderen Polynomen, sowie die Bestimmung der zugehörigen Fehler nach der Methode der kleinsten
Fehlerquadrate, wird mit SciPy 1.1.0\cite{scipy} durchgeführt.
