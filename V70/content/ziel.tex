\section{Zielsetzung}
\label{sec:Zielsetzung}
In diesem Versuch sollen die Grundlagen der Vakuumphysik und der Umgang mit den Komponenten der Vakuumtechnik erlernt werden.
Dazu wird ein Vakuumpumpstand mit verschiedenen Messgeräten und Ventilen aufgebaut.
An diesem wird dann das Saugvermögen zweier verschiedener Vakuumpumpen anhand der Evakuierungskurven und der Leckratenmessung bestimmt.\\
Ein einfaches Beispiel für die Nutzung von Vakua ist die Konservierung von verderblichen Produkten, da bei geringer Luftsauerstoffmenge Verwesungsprozesse nur stark eingeschränkt statt finden.
Desweiteren werden Vakua zur Wärmebehandlung von Metallen verwendet, da so das Oxidieren durch Sauerstoff verhindert werden kann. Ein Grobvakuum wird in der Industrie auch als Sauggreifer
verwendet um flächige Werkstücke zu greifen und zu transportieren\cite{anwendung}.
Eine weitere Anwendung von Vakua sind zum Beispiel Teilchenbeschleuniger wie der Large Hadron Collider am CERN, in welchem ein Ultrahochvakuum erzeugt wird, um Stoßprozesse der beschleunigten
Teilchen mit Restgasteilchen in der Vakuumkammer zu verhindern\cite{lhc}.