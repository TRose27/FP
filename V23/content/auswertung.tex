\section{Auswertung}
\label{sec:Auswertung}
\subsection{Partikel im periodischen Potenzial}
Die Spektren für die in \ref{tab:1} zu sehenden Rohrlängen sind in Abbildung \ref{fig:1} dargestellt.
Es ist zu beobachten, dass, bei gleicher Breite des Spektrums, mit steigender Rohrlänge die Anzahl an Resonanzen zunimmt, und der Abstand zwischen den Resonanzen abnimmt.
Aus den einzelnen Spektren wurden die Abstände der Resonanzen bestimmt. Diese werden gegen die inverse Rohrlänge aufgetragen und linear mit SciPy \cite{scipy} gefittet.
Dies ist in Abbildung \ref{fig:vschall} zu sehen. Die bestimmten Parameter sind:
\begin{align*}
  a &= \SI{168.47\pm0.59}{\metre\per\second} \\
  b &= \SI{8.76\pm3.42}{\hertz}.
\end{align*}
Gemäß der Resonanzbedingung \eqref{eqn:resonanz} lässt sich $a$ folgendermaßen identifizieren:
\begin{equation*}
  a=\frac{c}{2}
\end{equation*}
Somit beträgt die gemessene Schallgeschwindigkeit $c_\text{gemessen} = \SI{336.94\pm1.18}{\metre\per\second}$. Dies entspricht einer Abweichung von $\SI{1.8}{\percent}$ vom
Literaturwert, welcher $\SI{343}{\metre\per\second}$ bei $T=\SI{20}{\celsius}$ beträgt \cite{vschall}.
\begin{table}
  \centering
  \caption{Rohrlänge, inverse Rohrlänge und Abstände der Resonanzen}
  \label{tab:1}
  \begin{tabular}[t]{c c c}
   \toprule
    {d / $\si{\milli\metre}$} & {$\frac{1}{d}$ / $\si{\per\milli\metre}$} & {$\Delta f$ / $\si{\hertz}$} \\
     \midrule
     \csvreader[no head,
     late after line=\\,
     late after last line=\\\bottomrule]%
     {data/1tab.csv}{}%
     {$\num{\csvcoli}$ & $\num{\csvcolii}$ & $\num{\csvcoliii}$}%
   \end{tabular}
 \end{table}
\begin{figure}
  \centering
  \includegraphics[scale=0.7]{build/1.pdf}
  \caption{Es wurde für die angegeben Rohrlängen jeweils ein Spektrum im Bereich von $\SI{100}{\hertz}$ bis $\SI{10}{\kilo\hertz}$ gemessen und geplottet. Es ist zu sehen, dass die Anzahl der Resonanzstellen zunimmt.}
  \label{fig:1}
\end{figure}
\begin{figure}
  \centering
  \includegraphics{build/vschall.pdf}
  \caption{Die Abstände der Resonanzstellen $\Delta f$ wurde gegen die inverse Rohrlänge $\sfrac{1}{d}$ aufgetragen und linear gefittet. Aus der Steigung der Gerade lässt sich so die Schallgeschwindigkeit bestimmen.}
  \label{fig:vschall}
\end{figure}
\FloatBarrier

In Abbildung \ref{fig:2} ist ein $\SI{0.4}{\kilo\hertz}$ bis $\SI{12}{\kilo\hertz}$ Spektrum für eine Rohrlänge von $\SI{600}{\milli\metre}$ dargestellt.
Daraus wurden die Nummer der Resonanzstelle, sowie die dazugehörige Resonanzfrequenz bestimmt. Diese werden nun gegeneinander aufgetragen. So ergibt sich die ebenfalls in
\ref{fig:2} zu sehende Dispersionsrelation. Diese ist, da keine Störungen oder Unregelmäßigkeiten im Rohr vorliegen, linear. Die bestimmten Wertepaare befinden sich in Tabelle
\ref{tab:2}.
\begin{table}
  \centering
  \caption{Nummer der Resonanzstelle k und dazugehörige Frequenzen}
  \label{tab:2}
\begin{tabular}[t]{c c}
 \toprule
   {k} & {$f$ / $\si{\hertz}$} \\
   \midrule
   \csvreader[no head,
   late after line=\\,
   late after last line=\\\bottomrule,
  filter test={\ifnumless{\thecsvinputline}{20}}]%
   {data/2tab.csv}{}%
   {$\num{\csvcoli}$ & $\num{\csvcolii}$}%
 \end{tabular}
 \begin{tabular}[t]{c c}
  \toprule
    {k} & {$f$ / $\si{\hertz}$} \\
    \midrule
    \csvreader[no head,
    late after line=\\,
    filter test={\ifnumgreater{\thecsvinputline}{19}},
    late after last line=\\\bottomrule]%
    {data/2tab.csv}{}%
    {$\num{\csvcoli}$ & $\num{\csvcolii}$}%
  \end{tabular}
\end{table}
\begin{figure}
\centering
\includegraphics[scale=0.7]{build/2.pdf}
\caption{In a) ist das Spektrum von $\SI{0.4}{\kilo\hertz}$ bis $\SI{12}{\kilo\hertz}$ bei einer Rohrlänge von $\SI{600}{\milli\metre}$ zu sehen. In b) ist die dazugehörige lineare Dispersionsrelation dargestellt.}
\label{fig:2}
\end{figure}
\FloatBarrier

In Abbildung \ref{fig:3} sind die $\SI{0.4}{\kilo\hertz}$ bis $\SI{12}{\kilo\hertz}$ Spektren und Dispersionsrelationen für ein Rohr welches aus 12 $\SI{50}{\milli\metre}$
Röhren, zwischen welchen sich Blenden befinden, besteht. Die Messung wurde einmal mit $\SI{10}{\milli\metre}$, einmal mit $\SI{13}{\milli\metre}$ und einmal mit
$\SI{16}{\milli\metre}$ Blendendurchmesser durchgeführt. Aus den resultierenden Spektren wurden wieder die Nummern der Resonanzstellen und die Resonanzfrequenzen bestimmt und
geplottet. Die Messdaten befinden sich in Tablle \ref{tab:3}.

\begin{table}
  \centering
  \caption{Nummer der Resonanzstelle k und dazugehörige Frequenzen für verschiedene Blendendurchmesser}
  \label{tab:3}
\begin{tabular}[t]{c c}
 \toprule
 \multicolumn{2}{c}{$\SI{10}{\milli\metre}$} \\
 \cmidrule(lr){1-2}
   {k} & {$f$ / $\si{\hertz}$} \\
   \midrule
   \csvreader[no head,
   late after line=\\,
   late after last line=\\\bottomrule]%
   {data/3tab0.csv}{}%
   {$\num{\csvcoli}$ & $\num{\csvcolii}$}%
 \end{tabular}
 \begin{tabular}[t]{c c}
  \toprule
  \multicolumn{2}{c}{$\SI{13}{\milli\metre}$} \\
  \cmidrule(lr){1-2}
    {k} & {$f$ / $\si{\hertz}$} \\
    \midrule
    \csvreader[no head,
    late after line=\\,
    late after last line=\\\bottomrule]%
    {data/3tab1.csv}{}%
    {$\num{\csvcoli}$ & $\num{\csvcolii}$}%
  \end{tabular}
  \begin{tabular}[t]{c c}
   \toprule
   \multicolumn{2}{c}{$\SI{16}{\milli\metre}$} \\
   \cmidrule(lr){1-2}
     {k} & {$f$ / $\si{\hertz}$} \\
     \midrule
     \csvreader[no head,
     late after line=\\,
     late after last line=\\\bottomrule]%
     {data/3tab2.csv}{}%
     {$\num{\csvcoli}$ & $\num{\csvcolii}$}%
   \end{tabular}
\end{table}
\begin{figure}
\centering
\includegraphics[scale=0.8]{build/3.pdf}
\caption{In a) sind die Spektren und in b) die Dispersionsrelationen für die angegebenen Blendendurchmesser für einen Frequenzbereich von $\SI{0.4}{\kilo\hertz}$ bis $\SI{12}{\kilo\hertz}$ dargestellt. Dabei sind vom Blendendurchmesser abhängige Bandstrukturen zu erkennen.}
\label{fig:3}
\end{figure}

In den Spektren und Dispersionsrelationen sind Bandstrukturen zu erkennen, welche auf die eingebauten Blenden zurück zu führen sind. Es ist dabei zu beobachten, dass bei zunehmenden
Blendendurchmesser die Bandlücken kleiner und die Bänder größer werden. Des weiteren ist eine Verschiebung der Bänder zu niedrigeren Frequenzen zu beobachten. Außerdem sind bei größeren
Blenden mehr Resonanzen pro Band aufgetreten, was aber auf die größere Bandbreite zurückzuführen ist.
Unabhängig vom Blendendurchmesser ist zu beobachten, dass bei steigender Frequenz die Bandlücken größer und die Bandbreiten kleiner werden.
Aus den in \ref{fig:3} dargestellten Dispersionsrelationen wurden die spektralen Breiten der Bandlücken $\Delta f_\text{Lücke}$ und Bänder $\Delta f_\text{Band}$ bestimmt. Die Ergebnisse befinden sich in Tabelle \ref{tab:3band}.
Bei den Messdaten fällt auf, dass bei der $\SI{16}{\milli\metre}$ Blende das zweite Band breiter als das erste ist, was nicht zu den zuvor beobachteten Gesetzmaßigkeiten passt.
Dies geht wahrscheinlich darauf zurück, dass Teile des ersten Bandes nicht mehr im Messbereich liegen.

\begin{table}
  \centering
  \caption{Spektrale Breiten der Bänder und Bandlücken}
  \label{tab:3band}
   \noindent\makebox[\textwidth]{
\begin{tabular}[t]{c c c c c c c}
 \toprule
 &\multicolumn{2}{c}{$\SI{10}{\milli\metre}$}  &\multicolumn{2}{c}{$\SI{13}{\milli\metre}$}  &\multicolumn{2}{c}{$\SI{16}{\milli\metre}$} \\
 \cmidrule(lr){2-3}  \cmidrule(lr){4-5}  \cmidrule(lr){6-7}
   {Band} & {$\Delta f_\text{Band}$ / $\si{\hertz}$} & {$\Delta f_\text{Lücke}$ / $\si{\hertz}$} & {$\Delta f_\text{Band}$ / $\si{\hertz}$} & {$\Delta f_\text{Lücke}$ / $\si{\hertz}$} & {$\Delta f_\text{Band}$ / $\si{\hertz}$} & {$\Delta f_\text{Lücke}$ / $\si{\hertz}$}\\
   \midrule
    1 & 1140 & 1736 & 1403 & 1384 & 1649 & 986  \\
    2 & 745  & 2610 & 1228 & 2094 & 1753 & 1537 \\
    3 & 492  & 2921 & 793  & 2505 & 1263 & 2089 \\
    4 & 270  &      & 677  &      & 1014 &      \\
    \bottomrule
 \end{tabular}
 }
\end{table}
