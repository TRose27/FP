\section{Diskussion}
\label{sec:Diskussion}
Die aus dem Experiment bestimmte Schallgeschwindigkeit von $c = \SI{336,94\pm1,18}{\meter\per\second}$, mit einer Abweichung von $\SI{1,8}{\percent}$ vom Literaturwert,
lässt auf eine ziemlich präzise Messung schließen.\\
Die Abbildungen \ref{fig:1}, \ref{fig:2}, \ref{fig:6} und \ref{fig:7} zeigen gut die Analogie zwischen dem quantenmechanischen Potentialtopf und dem Schall in der Röhre,
welcher nur diskrete Werte als Eigenenergie zulässt. Die Resonanzfrequenzen sind ebenfalls diskret und die Anzahl nimmt mit zunehmender Länge der Röhre zu.\\
Die Abbildungen \ref{fig:3}, \ref{fig:8} und \ref{fig:9} zeigen wie sich eine Schallwelle bzw. die Wellenfunktion eines Teilchens verhält, wenn diese an einem hindernis
mit variierenden Wirkungsquerschnitt gestreut wird. Die Bandstrukturen welche aus der Streuung resultieren zeigen ganz gut die Energieniveaus des Elektrons.\\
In Abbildung \ref{fig:4} ist zu sehen, dass die Anzahl an Resonanzen und Amplituden, mit steigender Anzahl von Elementarzellen zunimmt.\\
Ein Molekül, bestehend aus zwei Atomen, wurde in Abbildung \ref{fig:10} mit alternierenden Blendendurchmesser simuliert. Mit einem Spektrum verglichen, indem der 
Blendendurchmesser konstant bleibt, zeigt sich die Aufteilung in Zwischenstufen deutlich.\\
Im Ganzen betrachte lässt sich sagen, dass die Nutzung von Schallwellen als Analogie zur Quantenmechanik durchaus funktioniert, vorallem um das Verhalten eines 
Elektron in einem Gitter oder Potentialtopf zu untersuchen.