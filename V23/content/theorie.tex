\section{Theorie}
\label{sec:Theorie}

\subsection{Resonanz}
Wenn in einem geschlossenen Rohr ein Schall erzeugt wird und dieser von der gegenüberliegenden Wand reflektiert wird, entsteht eine stehende Welle im Rohr.
Die hin- und zurücklaufende Wellen interferieren miteinander. Resonanz entsteht an den Stellen, wo die Wellen konstruktiv interferieren und die Resonanzbedingung 
\begin{equation}
    2L = \frac{nc}{f}
    \label{eqn:resonanz}
\end{equation}
erfüllt ist. Dabei ist $L$ die Länge des Rohrs, $c$ die Schallgeschwindigkeit, $f$ die Frequenz und $n$ eine natürliche Zahl.


\subsection{Gemeinsamkeiten von akustischen und quantenmechanischen Systemen}
Aus der linearen Euler Gleichung
\begin{equation*}
    \frac{\partial v}{\partial t} = -\frac{1}{\rho}\vec{\nabla}p, 
\end{equation*}
mit der Geschwindigkeit $v$ und der Dichte $\rho$, und der Kontinuitätsgleichung
\begin{equation*}
    \frac{\partial \rho}{\partial t} = -\rho \vec{\nabla} \cdot \vec{v}
\end{equation*}
folgt, dass sich eine eindimensionale Welle mit 
\begin{equation*}
    p(x) = p_0 \cos(kx-\omega t)
\end{equation*}
beschreiben lässt. In einer Röhre mit der Länge $L$ lässt sich mit den Randbedingungen $\dfrac{\partial p}{\partial x}(0)=0$ und $\dfrac{\partial p}{\partial x}(L)=0$
herleiten, dass $k=\dfrac{n\pi}{L}$ ist. Dabei ist $n$ eine ganzzahliger positive Zahl.\\
Analog kann ein Teilchen in einem Potentialtopf mit der Schrödingergleichung beschrieben werden:
\begin{equation*}
    E\psi(r) = -\frac{\hbar}{2m}\Delta\psi(r) + V(r)\psi(r).
\end{equation*}
In einem unendlich hohen Potentialtopf dessen Zwischenraum ein Potential von $V(r)=0$ besitzt, vereinfacht sich die Schrödingergleichung zu 
\begin{equation*}
    E\psi(r) = -\frac{\hbar}{2m}\Delta\psi(r).
\end{equation*}
Diese Gleichung kann für bestimmte Eigenwerte der Energie $E$ mit dem Ansatz
\begin{equation*}
    \psi(x) = A\sin(kx+\alpha)
\end{equation*}
gelöst werden. Aus den Randbedingungen folgt $\alpha=0$ und $k=\frac{n\pi}{L}$. In der Quantenmechanik ist die Energie immer mit der Frequenz verbunden:
\begin{equation*}
    E = hf = \hbar\omega.
\end{equation*}
Damit folg für die Eigenwerte der Energie 
\begin{equation*}
    E(k) = \frac{\hbar^2k^2}{2m} = \frac{\hbar^2n^2\pi^2}{2mL^2}.
\end{equation*}

Die Schallwelle und die Wellenfunktion lösen in beiden Fällen die Wellengleichung, welche ein delokalisiertes Objekt beschreibt. In beiden Fällen gibt es  
Eigenzustände und der Wellenvektor $k$ ist ebenfalls gleich.



\subsection{Unterschiede von akustischen und quantenmechanischen Systemen}
Als erstes wird im klassischen Fall die Amplitude von einem Mikrofon gemessen. Im quantenmechanischen Fall lässt sich jedoch aus dem Betragsquadrat nur die 
Wahrscheinlichkeit berechnen, ein Teilchen an einer bestimmten Position zu messen.\\
Desweiteren unterscheiden sich die Wellengleichungen in der zeitlichen Ableitung. Wärend die klassische Wellengleichung wegen der zweiten zeitlichen Ableitung 
periodische Lösungen besitzt, folgt dies im quantenmechanischen Fall aus der ersten Zeitableitung in Verbindung mit einem komplexen Phasenfaktor. Diese 
komplexwertige Funktionen können nicht direkt beobachtet oder gemessen werden, sondern nur das Betragsquadrat.\\
Die Dispersionsrealationen unterscheiden sich ebenfalls. Im klassischen Fall ist diese linear, wobei sie in der Quantenmechanik einen parabolischen Verlauf folgt.
Die Folge daraus sind unterschiedliche Gruppen- und Phasengeschwindigkeiten.\\
Zuletzt fordert die Quantenmechanik das die Wellenfunktion an den Rändern des unendllichen Potentialtopfs verschwindet, wobei im klassischen Fall der Druck dort 
einen Scchwingungsbauch vorweist.


\subsection{Analogien zur Festkörperphysik}
Mittels dem einsetzen von Blenden zwischen den einzelnen Röhrenbauteile, welche den Schall streuen, kann eine am Atom gestreute Elektronenwelle simuliert werden.
Dies führt zu neuen Resonanzfrequenzen, welche zur Bandstrukturen führen, wobei die Blendgröße ein Maß für den Wirkungsquerschnitt ist. Die Bragg Bedingung erfüllt wird,
also 
\begin{equation*}
    n\lambda = 2a
\end{equation*}
gilt. Dabei ist $n$ wieder eine natürliche Zahl und $a$ ist der Abstand der reflektierenden Ebenen. In diesem Versuch stellen die Blenden die reflektierenden Ebenen dar 
und der Abstand ist die Röhrenlänge.