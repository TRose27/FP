\section{Durchführung}
\label{sec:Durchführung}
Der allgemeine Versuchsaufbau besteht aus einer Röhre, welche aus Rohrstücken mit unterschiedlichen Längen zusammengesetzt wird. Am einem Ende der Röhre ist ein 
Mikrofon und am anderen Ende ein Lautsprecher eingebaut. Zum erzeugen und empfangen des Tonsignals, wird ein Computer verwendet, welcher über einen Controller mit 
der Röhre verbunden ist. Das gemessene Signal wird mit dem Programm "SpectrumSLC.exe" aufgenommen und von diesem als Plot dargestellt, welcher mit den Messwerten 
abgespeichert wird. Für die einzelnen Versuchsteile wird die Röhre aus unterschiedlichen Stücken zusammengesetzt. Zur verfügung stehen Rohrstücke mit der Länge von 
$\SI{12,5}{\milli\meter}$, $\SI{50}{\milli\meter}$ und $\SI{75}{\milli\meter}$. Desweiteren werden drei unterschiedliche Blenden verwendet, welche Durchmesser von 
$\SI{10}{\milli\meter}$, $\SI{13}{\milli\meter}$ und $\SI{16}{\milli\meter}$ haben.\\
Als erstes wird der Lautsprecher und das Mikrofon des Computers so eingestellt, dass es ein maximales Signal gibt, aber das Mikrofon nicht übersättigt. Dies lässt sich 
zum einem mit den Computereinstellung regeln und ebenfalls mit dem Abschwächer des Controllers. Wenn zwischen Mikrofon und Lautsprecher ein $\SI{75}{\milli\meter}$
langes Rohrstück eingebaut wird und ein Signal von $\SI{100}{\hertz}$ bis $\SI{10000}{\hertz}$ gesendet wird, zeigt das benutzte Computerprogramm selbst an wenn das 
das Mikrofon übersättigt ist. Durch ändern der oben genannten Einstellung, wird die Messung wiederholt bis das Mikrofon nicht mehr übersättigt ist. Anschließend können 
die Messungen für das Experiment beginnen.\\
Als erstes wird im Frequenzbereich von $\SI{10}{\hertz}$ bis $\SI{10000}{\hertz}$ in $\SI{10}{\hertz}$ Schritten das Spektrum von einem $\SI{75}{\milli\meter}$ Rohrstück aufgenommen.
Dies wird wiederholt und jedes mal ein Rohrstück der selben Länge hinzugefügt, bis insgesammt acht $\SI{75}{\milli\meter}$ Rohrstücke verbaut sind.\\
Für die nächsten vier Versuchsabschnitte wird der Frequenzbereich von $\SI{0,4}{\kilo\hertz}$ bis $\SI{12}{\kilo\hertz}$ eingestellt. In diesem Bereich wird zuerst 
das Spektrum für zwölf $\SI{50}{\milli\meter}$ Rohrstücke aufgenommen. Für acht Rohrstücke dieser Länge wird das Spektrum als nächstes mit den Blenden gemessen.
Eine Blendengröße wird jeweils zwischen die einzelnen Rohrstücke eingebaut. Dies wird für alle drei Blendengrößen durchgeführt. Als nächstes wird das Spektrum von 
zwölf und zehn $\SI{50}{\milli\meter}$ Rohrstücken aufgenommen, mit den $\SI{16}{\milli\meter}$ Blenden zwischen den Bauteilen. Als letztes in diesem Versuchsteil wird 
das Spektrum von acht $\SI{75}{\milli\meter}$ Stücken mit den $\SI{16}{\milli\meter}$ Blenden gemessen.\\
Für die ersten beiden Messungen im zweiten Versuchsteil, wird ein Frequenzbereich von $\SI{0,4}{\kilo\hertz}$ bis $\SI{22}{\kilo\hertz}$ eingestellt.
Als erstes wird das Spektrum eines einzelnen $\SI{50}{\milli\meter}$ Stückes aufgenommen. Dies wird für ein $\SI{75}{\milli\meter}$ Stück wiederholt.
Für die nächsten vier Messungen wird der Maximalwert des Frequenzbereiches wieder auf $\SI{12}{\kilo\hertz}$ gestellt.
Es wird für zwei $\SI{50}{\milli\meter}$ Stücke, mit einer Blende dazwischen, das Spektrum aufgenommen. Dies wird für alle drei Blendengrößen durchgeführt.
In der nächsten Messung wird jeweils für drei, vier und sechs Einheitszellen die Spektren drei mal gemessen, jeweils mit einer anderen Blendengröße zwischen den 
eizelnen Rohrstücken. Als nächstes wird das Spektrum für zwölf $\SI{50}{\milli\meter}$ Stücke gemessen, welche abwechselnd mit $\SI{13}{\milli\metre}$ und 
$\SI{16}{\milli\meter}$ Blenden getrennt sind. Für letzte Messung in diesem Frequenzbereich, werden fünf Einheitszellen aufgebaut, welche jeweils aus einem 
$\SI{50}{\milli\meter}$ Stück, einer $\SI{16}{\milli\meter}$ Blende, einem $\SI{75}{\milli\meter}$ Rohrstück und noch einer $\SI{16}{\milli\meter}$ Blende bestehen.\\
Die letzte Messung wird in einem Frequenzbereich von $\SI{0,4}{\kilo\hertz}$ bis $\SI{6}{\kilo\hertz}$ durchgeführt. Es werden zwölf $\SI{50}{\milli\meter}$
Rohrstücke aufgebaut, jeweils mit einer $\SI{16}{\milli\meter}$ Blende dazwischen. Nun wird das zweite Rohrstück von der Lautsprecherseite aus durch ein 
$\SI{75}{\milli\meter}$ Rohrstück ersetzt und das Spektrum gemessen. Danach werden die Rohrstücke wieder ausgetauscht und das siebte Rohrstück von der Lautsprecherseite aus 
durch ein $\SI{75}{\milli\meter}$ Stück ersetzt. Dies wird nochmal für das dritte Rohrstück von der Mikrofonseite aus wiederholt. An dieser Stelle werden ebenfalls die 
Spektren für andere Rohrlängen als Defekt gemessen. Die Rohrlängen betragen $\SI{25}{\milli\meter}$, $\SI{37,5}{\milli\meter}$ und $\SI{62,5}{\milli\meter}$.